\documentclass[english, 11 pt, class=article, crop=false]{standalone}
%\documentclass[english, 11 pt]{report}
\usepackage[T1]{fontenc}
\usepackage[utf8]{luainputenc}
\usepackage{babel}
\usepackage[hidelinks, bookmarks]{hyperref}
\usepackage{geometry}
\geometry{verbose,tmargin=1cm,bmargin=3cm,lmargin=4cm,rmargin=4cm,headheight=3cm,headsep=1cm,footskip=1cm}
\setlength{\parindent}{0bp}
\usepackage{amsmath}
\usepackage{amssymb}
\usepackage{esint}
\usepackage{import}
\usepackage[subpreambles=false]{standalone}
%\makeatletter
\addto\captionsenglish{\renewcommand{\chaptername}{Kapittel}}
\makeatother
\usepackage{tocloft}
\addto\captionsenglish{\renewcommand{\contentsname}{Innhold}}
\usepackage{graphicx}
\usepackage{placeins}
\raggedbottom
\usepackage{calc}
\usepackage{cancel}
\makeatletter
\usepackage{color}
\definecolor{shadecolor}{rgb}{0.105469, 0.613281, 1}
\usepackage{framed}
\usepackage{wrapfig}
\usepackage{bm}
\usepackage{ntheorem}

\usepackage{ragged2e}
\RaggedRight
\raggedbottom
\frenchspacing

\newcounter{lign}[section]
\newenvironment{lign}[1][]{\Large \refstepcounter{lign} \large
	\textbf{\thelign #1} \rmfamily}{\par\medskip}
\numberwithin{lign}{section}
\numberwithin{equation}{section}
\usepackage{xcolor}
\usepackage{icomma}
\usepackage{mathtools}
\usepackage{lmodern} % load a font with all the characters
\usepackage{xr-hyper}
\makeatother
\usepackage[many]{tcolorbox}

%\setlength{\parskip}{\medskipamount}
\newcommand{\parskiplength}{11pt}
%\setlength{\parskip}{0 pt}
\newcommand\eks[2][]{\begin{tcolorbox}[enhanced jigsaw,boxrule=0.3 mm, arc=0mm,breakable,colback=green!30] {\large \textbf{Eksempel #1} \vspace{\parskiplength}\\} #2 \vspace{1pt} \end{tcolorbox}\vspace{1pt}}

\newcommand\fref[2][]{\hyperref[#2]{\textsl{Figur \ref*{#2}#1}}}
\newcommand{\hr}[2]{\hyperref[#2]{\color{blue}\textsl{#1}}}

\newcommand\rgg[2][]{\begin{tcolorbox}[boxrule=0.3 mm, arc=0mm,colback=orange!55] #2 \vspace{1pt} \end{tcolorbox}\vspace{-2pt}}
\newcommand\alg[1]{\begin{align*} #1 \end{align*}}
\newcommand\algv[1]{\vspace{-11 pt} \begin{align*} #1 \end{align*}}
\newcommand\vs{\vspace{-11 pt}}
\newcommand\g[1]{\begin{center} {\tt #1}  \end{center}}
\newcommand\gv[1]{\begin{center} \vspace{-22 pt} {\tt #1} \vspace{-11 pt} \end{center}}
%\addto\captionsenglish{\renewcommand{\contentsname}{Løsningsforslag tentamen R2 H2015}}

% Farger
\colorlet{shadecolor}{blue!30} 

% Figur
\usepackage{float}
\usepackage{subfig}
\captionsetup[subfigure]{labelformat=empty}
\usepackage{esvect}

\newcommand\sv{\textbf{Svar:} \vspace{5 pt} \\}

%Tableofconents
\renewcommand{\cfttoctitlefont}{\Large\bfseries}
\setlength{\cftsubsecindent}{2 cm}
\newcommand\tocskip{6 pt}
\setlength{\cftaftertoctitleskip}{30 pt}
\setlength{\cftbeforesecskip}{\tocskip}
%\setlength{\cftbeforesubsecskip}{\tocskip}

%Footnote:
\usepackage[bottom, hang, flushmargin]{footmisc}
\usepackage{perpage} 
\MakePerPage{footnote}
\addtolength{\footnotesep}{2mm}
\renewcommand{\thefootnote}{\arabic{footnote}}
\renewcommand\footnoterule{\rule{\linewidth}{0.4pt}}

%asin, atan, acos
\DeclareMathOperator{\atan}{atan}
\DeclareMathOperator{\acos}{acos}
\DeclareMathOperator{\asin}{asin}

%Tabell
\addto\captionsenglish{\renewcommand{\tablename}{Figur}}

% Figur
\usepackage[font=footnotesize,labelfont=sl]{caption}
\addto\captionsenglish{\renewcommand{\figurename}{Figur}}

% Figurer
\newcommand\scr[1]{/home/sindre/R/scr/#1}
\newcommand\asym[1]{/home/sindre/R/asymptote/#1}

%Toc for seksjoner
\newcommand\tsec[1]{\phantomsection\addcontentsline{toc}{section}{#1}
	\section*{#1}}
%\newcommand\tssec[1]{\subsection*{#1}\addcontentsline{toc}{subsection}{#1}}
\newcommand\tssec[1]{\subsection*{#1}}
% GeoGebra
\newcommand{\cms}[2]{{\tt #1( #2 )}}
\newcommand{\cm}[2]{{\large \tt #1( #2 )} \gvs \\}
\newcommand{\cmc}[2]{{\large \tt #1( #2 )} \large (CAS)  \gvs \\ \normalsize}
\newcommand{\cmk}[2]{{\large \tt #1( #2 )} \large (Inntastingsfelt)  \gvs \\ \normalsize}

\newcommand\gvs{\vspace{11 pt}}

\newcommand\vsk{\vspace{11 pt}}
\newcommand{\merk}{\vsk \textsl{Merk}: }
\newcommand{\fig}[1]{
\begin{figure}
	\centering
	\includegraphics[scale=0.5]{fig/#1}
\end{figure}
}
\newcommand{\figc}[1]{
		\centering
		\includegraphics[scale=0.5]{fig/#1}
}

% Opg
%\newcommand{\opgt}{\phantomsection \addcontentsline{toc}{section}{Oppgaver} \section*{Oppgaver for kapittel \thechapter}}
\newcounter{opg}
\numberwithin{opg}{section}

\newcommand{\opl}[1]{\vspace{15pt} \refstepcounter{opg} \textbf{\theopg} \vspace{2 pt} \label{#1} \\}


 
\begin{document}
\eqlen	
%\setcounter{chapter}{4}	
%\tableofcontents
%\chapter{Derivasjon og funksjonsdrøfting}
\textbf{Mål for opplæringen er at eleven skal kunne}
\subimport{}{rg}
\begin{itemize}
	\item derivere sentrale funksjoner og bruke førstederiverte og andrederiverte til å drøfte slike funksjoner
\end{itemize}

\newpage


\begin{comment}
En funksjon $ f(u(x)) $ er avhengig av funksjonen $ u $, som igjen er avhengig av variabelen $ x $. Dette betyr at $ f $ kan deriveres med hensyn på enten $ u $ eller $ x $. Så lenge $ x $ er den uavhengige variabelen vil $ f'$ fremdeles bety $ f'(x) $, mens $ f'(u) $ altså indikerer $ f $ \textsl{derivert med hensyn på} $ u $.
\end{comment}
\section{Derivasjonsregler}
I tidligere matematikkkurs lærte du å derivere\index{derivasjon} grunnleggende funksjoner og sammensatte funksjoner. Vi skal ta med oss en liten repetisjon av derivasjonsreglene og i tillegg presentere den deriverte av $ \sin x $, $ \cos x $ og $ \tan x $. Men først må vi ha en liten redegjøring for føringen av funksjoner og deres deriverte:\vsk

For en funksjon $ f(x) $ vil $ f'(x) $ betegne $ f $ \textsl{derivert med hensyn på} $ x $. Hvis det på forhånd er etablert at $ f $ er en funksjon av $ x $, vil vi skrive $ f'(x) $ bare som $ f' $. \regv
\rg[Definisjon av den deriverte]{
For en deriverbar funksjon $ f(x) $ er den deriverte med hensyn på $ x $ definert som
\begin{equation}
f'(x)=\lim\limits_{\Delta x\to 0}\frac{f(x+\Delta x)-f(x)}{\Delta x} \label{defder}
\end{equation}\vs
}
\dfdx
\kjerne\newpage
\eks{
	Finn $ f' $ når  $ f(x)=(x^2+x)^2 $
	\\
	
	\sv 
	Vi setter $ u(x)=x^2+x$, og får at
	\begin{align*}
	& g(u)=u^2 \\
	& g'(u)=2u \\
	& u' = 2x + 1
	\end{align*}
	Altså blir
	\algv{
		f' &= g'(u)u' \\
		&= 2u(2x+1)\\
		&= 2(x^2+x)(2x+1) 
	}\vds
}
\prreg
\eks{Gitt funksjonen $ f(t)=t^2e^t $. Finn $ f'$. \vsk\\
	
	\sv
	Vi setter $ u(t)=t^2 $ og $ v(t)=e^t $, og får at
	\alg{
	u'&=2t\\
	v' &= e^t	
}
Videre er da\vs	
	\alg{f' &= (uv)' \\
		&= u'v+uv' \\ 
		&= 2te^t+t^2e^t \\
		&= te^t(2+t)}\vds
}
\rg[Divisjonsregelen ved derivasjon]{
For funksjonen $ f(x)=\dfrac{u(x)}{v(x)} $ har vi at
\begin{equation}\label{divreg}
f'=\frac{u'v-uv'}{v^2}
\end{equation}\vsb
}
\eks{
	Gitt funskjonen $ f(x)=\frac{x^2}{\sin x} $. Finn $ f' $.

\sv 

Vi setter $ {u(x)=x^2 }$ og $ {v(x)=\sin x} $, og får da at
\alg{
u' &= 2x \\
v' &= \cos x
}
Videre er da
\alg{
f' &= \left(\frac{u}{v}\right)' \br
&= \frac{u'v-uv'}{v^2} \br
&= \frac{(x^2)'\sin x-x^2(\sin x)'}{\sin^2 x} \br
&= \frac{2x\sin x - x^2\cos x}{\sin^2 x} \br
&= x\sin^{-2}(2\sin x-x\cos x)
}
\textit{Merk}: \eqref{divreg} er bare en utvidelse av \eqref{prreg} kombinert med potensregelen $ \frac{1}{a}=a^{-1} $. Uten å bruke \eqref{divreg} kunne vi derfor løst oppgaven slik\footnote{Minner igjen om at $ \sin^{-1} x$ i denne boka er det samme som $ \frac{1}{\sin x} $, mens uttrykket i andre tekster kan være samsvarende med $ \asin x $.}:\vsk

Vi observerer at
\[ f(x) = x^2 \sin^{-1} x \]
Av \eqref{prreg} er da
\alg{
f' &=\left(x^2\right)'\sin^{-1} x + x^2\left(\sin^{-1} x\right)' \\
&= 2x \sin^{-1}x - x^2 \sin^{-2} \cos x \\
&= x\sin^{-1} x (2 x-x\sin^{-1}\cos x)
}
I derivasjonen av $ \sin^{-1} x$ har vi brukt kjerneregelen. Med litt omskriving vil du finne at det endelige svaret er ekvivalent med det vi fikk da vi brukte divisjonsregelen.
}
\section{Andrederiverttesten}
Trolig er du også kjent med å finne maksimum\footnote{Vi minner igjen om at en utfyllende liste over punkt på en graf er å finne i \hrv{pktpgrf}.} og minimum\index{maksimum}\index{minimum}\footnote{Maksimum og minimum blir også kalt \textit{maksimumsverdier} og \textit{minimumsverdier}.} til en funksjon $ f $ ved å studere $ f'$ via et fortegnsskjema, men ofte er \textit{andrederiverttesten}\index{andrederiverttest} mindre tidkrevende:\regv
\anddert
\newpage
\eks[1]{\label{maxmine}
Gitt funksjonen
\[ f(x)=x^3-3x^2\quad,\quad x\in[-2, 3] \]
\textbf{a)} Finn alle lokale maksimum og minimum for $f$.\os

\textbf{b)} Finn maksimal- og minimalverdien til $ f $.

\sv
\textbf{a)} Vi starter med å finne punktene hvor $ {f'(x)=0 }$:
\alg{
	f'(x)&= 0 \\
	3x^2-6x &= 0 \\
	3x(x-2) &= 0
	}
$ f'(x) $ er altså 0 for ${ x=0}  $ eller $ {x=2 }$. Videre finner vi at
\[ f''(x)=6x-6 \]
og at
\algv{f''(0)&=-6 \\ f''(2)&=6}
Av andrederiverttesten er da $ {x=0 }$  et lokalt maksimum og ${ x=2} $ et lokalt minimum for $ f $.\vsk

\textbf{b)} $ f $-verdiene for de to lokale ekstremalpunktene vi fant i a) er
\alg{f(0)&= 0 \\
	f(2)&=-4
	}
Men vi må ikke glemme å sjekke endepunktene til $ f $:
\alg{
	f(-2) &= -20 \\
	f(3) &= 0	
	}
Altså er $- 20 $ minimumsverdien til $ f $, mens 0 er maksimumsverdien.
	}
\eks[2]{
Gitt funksjonen 
\[ f(x)=\cos x \quad,\quad x\in[0, \pi]\]
Finn lokale maksimum og minimum for $ f $.\\

\sv
Vi har at
\algv{
f'(x)&= -\sin x \br
f''(x)&= -\cos x}
Dette betyr at ${ f'=0} $ for $ {x\in\lbrace{0, \pi\rbrace}} $ og at $ f''(\pi)=-f''(0)=1 $.  Men siden ${ x=0} $ og ${ x=\pi} $ er endepunkter for $ f $, er ikke $ f' $ kontinuerlig \textit{omkring} disse verdiene, dermed har $ f $ ingen lokale maksumim eller miniumum. Istedenfor er $ {f(0)=1} $ absolutt maksimum og $ {f(\pi)=-1} $ absolutt minimum.
}
\section{Den antideriverte}\index{antiderivert}
Vi skal nå se på en definisjon som kan virke veldig triviell, men som viser seg å være en viktig brikke når vi i neste kapittel skal studere \textit{integrasjon}. \vsk

La oss starte med å se på funksjonen $ {f(x)=x^2}  $.
Å derivere $ f $ mhp. $ x $ byr på få problemer:
\[ f'=2x \]
Hva nå med den deriverte av $ {g(x)=x^2+1} $? Svaret blir det samme som for $ f'$:
\[ g'=2x \]
Allerede nå innser vi at det finnes en haug av funksjoner, rett og slett uendelig mange, som har $ 2x $ som sin deriverte. Tiden er derfor inne for å lage en samlebetegnelse for alle funksjoner med samme deriverte:\regv
\antider
\eks{
	Undersøk om følgende funksjoner er en antiderivert til ${f(x)=2x+e^x } $:
	\alg{
		g(x)&=x^2 +e^x  \\[5 pt]
		h(x)&=x^2 +e^{2x}\\[5 pt]
		k(x)&=x^2+e^x+4 
	}\vds
	
	\sv
	Vi finner de deriverte av $ g $, $ h $ og $ k $:
	\alg{
		g'(x)&=2x+e^x  \os 
		h'(x)&=2x+2e^{2x} \os
		k(x)&=2x+e^x
	}
	Siden $ {g'(x)=k'(x)=f(x)} $, mens $ { h'(x)\neq f(x)} $, er bare $ g(x) $ og $ k(x) $ en antiderivert til $ f $.
}
\newpage
\tsec{Forklaringer}
\subsection*{Derivasjonsregler}
Vi skal nøye oss med å finne uttrykket for den deriverte av de tre uttrykkene som ikke ble gitt i R1, nemlig $ \cos x, \sin x $ og $ \tan x $.\vsk

{\boldmath $ (\cos x)'=-\sin x $}\bs
Vi skal her anvende de to ligningene (se \hrv{sinxcosxlim})
\alg{&\lim\limits_{x\to0} \frac{\sin x}{x}=1 \tag{I} \label{Isinxx}\br
&\lim\limits_{x\to0} \frac{\cos x-1}{x}=0\tag{II} \label{Icosxx}	
}
Per definisjon (se \eqref{defder}) er $ (\cos x)' $ gitt som
\[ (\cos x)'= \lim\limits_{\Delta x\to 0}\frac{\cos(x+\Delta x)-\cos x}{\Delta x} \]
Ved (\ref{cu-v}) kan vi skrive
\alg{
 \lim\limits_{\Delta x\to 0}\frac{\cos(x+\Delta x)-\cos x}{\Delta x}&=\lim\limits_{\Delta x\to 0} \frac{\cos x \cos (\Delta x)-\sin x\sin(\Delta x)-\cos x}{\Delta x} \\
 &=\lim\limits_{\Delta x\to 0}\frac{( \cos(\Delta x)-1)\cos x-\sin x\sin(\Delta x)}{\Delta x} \\
 &= \lim\limits_{\Delta x\to 0}\frac{\cos(\Delta x)-1)}{\Delta x}\cos x- \lim\limits_{\Delta x\to 0} \frac{\sin(\Delta x)}{\Delta x}\sin x \\
 &= 0-1\cdot\sin x \\
 &= -\sin x
}
Mellom tredje og fjerde linje i likningen over brukte vi (\ref{Isinxx}) og (\ref{Icosxx}). \vsk

{\boldmath $ (\sin x)'=\cos x $}\bs
Av (\ref{cossomsin}), (\ref{sinsomcos}) og (\ref{-cosxcosx}) har vi at
\alg{
\sin x &= \cos\left(x-\frac{\pi}{2}\right) \br
\sin\left(\frac{\pi}{2}-x\right)&= \cos x
}
Bruker vi det faktum at $ (\cos x)'=-\sin x $, i kombinasjon med kjerneregelen, får vi at
\algv{
(\sin x)' &= \left(\cos\left(x-\frac{\pi}{2}\right)\right)' \\
&= -\sin\left(x-\frac{\pi}{2}\right)\cdot 1 \\
&= \sin\left(\frac{\pi}{2}-x\right) \\
&= \cos x
}\vsk

{\boldmath $ (\tan x)'=\frac{1}{\cos^2 x} $}\bs
Av kjerneregelen og produktregelen ved derivasjon (se (\ref{kjerne} og (\ref{prreg})) er
\alg{
	(\tan x)' &= \left(\sin x \cos^{-1} x\right)' \\
&= \cos x \cos^{-1}x +\sin x\left(\cos^{-1}\right)'\\
&= 1 +\sin x(-\cos^{-2}x)(-\sin x) \\
&= 1+\tan^2 x\\
&= \frac{\cos^2 x+\sin^2 x}{\cos^2 x} \tag{$ \cos^2 x+\sin^2 x=1 $} \\
&= \frac{1}{\cos^2 x}
}

\subsection*{Andrederiverttesten}
Av definisjonen for den deriverte har vi at
\[ f''(c)=\lim\limits_{\Delta x \to 0}\frac{f'(c+\Delta x)-f'(c)}{\Delta x} \]
Når $ f'(c)=0$, er
\[ f''(c)=\frac{f'(c+\Delta x)}{\Delta x} \]
Når $ f''(c)<0 $, betyr dette at
\[\lim\limits_{\Delta x \to 0} \frac{f'(c+\Delta x)}{\Delta x}<0 \]
Altså må $ f'(c+\Delta x) $ være positiv når $ \Delta x $ nærmer seg 0 fra negativ side av tallinjen og negativ når $ \Delta x $ nærmer seg 0 fra positiv side. Dermed skifter $ f' $ fortegn i $ c $, som da må være et maksimalpunkt for $ f $. Tilsvarende må $ c $ være et minimumspunkt for $ f $ hvis $ {f(c)=0} $ og $ {f''(c)<0} $.
\begin{comment}
Kommentering av ligninger.
\begin{flalign}
& & (a+b)^2&=a^2+2ab+b^2 && \llap{Right justify me!}
\end{flalign}
\subsection*{Infleksjonspunkt}
For å argumentere for at $ f $ går fra konkav/konveks til konveks/konkav i et vendepunkt skal vi bruke følgende teorem:

\textsl{Hvis $ f' $ er økende på et intervall, er $ f $ konveks på dette intervallet. Hvis $ f' $ er synkende på et intervall, er $ f $ konkav på dette intervallet.}

Konkave og konvekse funksjoner er en marginal del av R2-faget, derfor nøyer vi oss med grafen til $ f(x)=\sin x $ og $ f'(x) $ som overbevisning om teoremets gyldighet:
\begin{figure}
\centering
\includegraphics[]{\asym{konk2}}
\caption{Grafen til $ f(x)=\sin x $ og $ f'(x)=\cos x $. $ f $ er konkav i intervallet til venstre og konveks i intervallet til høyre.}
\end{figure}
Siden $ f' $ er voksende når $ f''>0 $ og synkende når $ f''<0 $, følger det direkte av teoremet over at $ f $ går fra konveks til konkav når $ f'' $ går fra positiv til negativ, og fra konkav til konveks når $ f'' $ går fra negativ til positiv. For kontinuerlig funksjoner betyr dette at $ f''=0 $ i et infleksjonspunkt, men legg merke til at $ f''(c)=0 $ ikke nødvendigvis betyr at $ f'' $ skifter fortegn i $ c $.


\end{comment}
\end{document}

