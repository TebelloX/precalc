\documentclass[english, 11 pt, class=article, crop=false]{standalone}
%\documentclass[english, 11 pt]{report}
\usepackage[T1]{fontenc}
\usepackage[utf8]{luainputenc}
\usepackage{babel}
\usepackage{xr-hyper}

\usepackage{geometry}
%\geometry{verbose,margin=1.5cm,bmargin=2.5cm,lmargin=4.5cm,rmargin=4.5cm,headheight=10cm,headsep=1cm,footskip=1cm}
\geometry{verbose,paperwidth=16.1 cm, paperheight=24 cm, inner=2.3cm, outer=1.8 cm, bmargin=2cm, tmargin=1.8cm}
\setlength{\parindent}{0bp}
\usepackage{amsmath}
\usepackage{amssymb}
\usepackage{esint}
\usepackage{import}
\usepackage[subpreambles=false]{standalone}
%\makeatletter
\usepackage{tocloft}
\usepackage{graphicx}
\usepackage{placeins}
\usepackage{calc}
\usepackage{cancel}
\usepackage{color}
\definecolor{shadecolor}{rgb}{0.105469, 0.613281, 1}
\usepackage{framed}
\usepackage{wrapfig}
\usepackage{bm}
\usepackage{ntheorem}
\usepackage{ragged2e}
\RaggedRight
\raggedbottom
%\flushbottom
\frenchspacing

\newcounter{lign}[chapter]
\newenvironment{lign}[1][]{\Large \refstepcounter{lign} \large
	\textbf{\thelign #1} \rmfamily}{\par\medskip}
\numberwithin{lign}{chapter}
\numberwithin{equation}{chapter}
\usepackage{xcolor}
\usepackage{icomma}
\usepackage{mathtools}
\usepackage{lmodern} % load a font with all the characters
\makeatother
\usepackage[many]{tcolorbox}

\newcommand{\parskiplength}{11pt}
%\setlength{\parskip}{\parskiplength}

\newcommand\eks[2][]{\begin{tcolorbox}[arc=0mm,enhanced jigsaw,breakable,colback=green!8,boxrule=0.3 mm] {\large \textbf{Eksempel #1} \vspace{5 pt}\newline} #2 \vspace{1pt} \end{tcolorbox}\vspace{-4pt}}
\newcommand\tit[2][]{\large \textbf{#2 - Eksempel #1} \\}
\newcommand\lig[1]{\begin{lign} \large \normalfont #1 \end{lign}}
\newcommand\hrds[1]{\hyperref[#1]{\textsl{delseksjon \ref*{#1}}}}
\newcommand\hr[2][]{\hyperref[#2]{\textsl{#1}}}
\newcommand\hrs[2][]{\hyperref[#2]{\textsl{\textsl{#1} \ref*{#2}}}}
\newcommand\hrv[1]{\hyperref[#1]{\textsl{\textsl{vedlegg} \ref*{#1}}}}
\newcommand\fref[2][]{\hyperref[#2]{\textsl{figur \ref*{#2}#1}}}
\newcommand\hrss[2][]{\hyperref[#2]{\textsl{#1}}}
\newcommand\ligg[1]{{\large \normalfont \textbf{#1}} \vspace{5 pt}\\}
%\newcommand\rg[2][]{\begin{tcolorbox}[colback=blue!15] \ligg{#1} #2  %\end{tcolorbox}}
\newcommand\rg[2][]{\begin{tcolorbox}[arc=0mm,colback=blue!5,enhanced jigsaw,breakable, boxrule=0.3 mm]{\large \normalfont \textbf{#1} \vspace{5 pt}\newline} #2 \vspace{1pt} \end{tcolorbox}\vspace{-4pt}}
\newcommand\rgg[2][]{\begin{tcolorbox}[colback=orange!55] #2 \vspace{1pt} \end{tcolorbox}\vspace{-9pt}}
\newcommand\alg[1]{\begin{align*} #1 \end{align*}}
\newcommand{\algv}[1]{\vspace{-9 pt} \begin{align*} #1 \end{align*}}
\newcommand{\algb}[1]{\vspace{-9 pt} \begin{align*} #1 \end{align*}}
\newcommand\vs{\vspace{-\parskiplength}}
\newcommand\vsb{\vspace{-13pt}}
\newcommand\vds{\vs\vs}
\newcommand\g[1]{\begin{center} \vspace{-11 pt} {\tt #1} \vspace{-11 pt} \end{center}}
\newcommand\gv[1]{\begin{center} \vspace{-22 pt} {\tt #1} \vspace{-11 pt} \end{center}}
\newcommand\vsk{\vspace{11 pt}}
%\addto\captionsenglish{\renewcommand{\contentsname}{Løsningsforslag tentamen R2 H2015}}

% Farger
%\pagecolor{yellow!3}
\colorlet{shadecolor}{blue!30} 

% Figur
\usepackage{float}
\usepackage{subfig}
\captionsetup[subfigure]{labelformat=empty}
\newcommand{\fig}[2]{\begin{figure}
		\centering
		\includegraphics[]{\asym{#1}}
		\caption{#2}
\end{figure}}
\newcommand{\net}[2]{{\color{blue}\href{#1}{#2}}}

\usepackage{esvect}
\usepackage[font=footnotesize,labelfont=sl]{caption}
\addto\captionsenglish{\renewcommand{\figurename}{Figur}}

\newcommand{\sss}[1]{\subsection*{#1}   \addcontentsline{toc}{subsection}{#1}}
\newcommand\sv{\vsk \textbf{Svar:} \vspace{4 pt}\\}
\newcommand{\bs}{\\[4 pt]}
\newcommand{\os}{\\[4 pt]}
%Tableofconents
%\setlength{\cftsubsecindent}{1 cm}
\renewcommand{\cfttoctitlefont}{\Large\bfseries}
\setlength{\cftaftertoctitleskip}{0 pt}
\setlength{\cftbeforetoctitleskip}{0 pt}
\renewcommand\cftchapfont{\footnotesize\bfseries}
\renewcommand\cftsecfont{\footnotesize}
\renewcommand\cftsubsecfont{\footnotesize}
\addto\captionsenglish{\renewcommand{\contentsname}{Innhold}}
\addto\captionsenglish{\renewcommand{\chaptername}{Kapittel}}
\newcommand\tocskip{3 pt}
\setlength{\cftbeforechapskip}{12 pt}
\setlength{\cftbeforesecskip}{\tocskip}
\setlength{\cftbeforesubsecskip}{\tocskip}

%Seksjoner
\usepackage{titlesec}
\titleformat{\chapter}[display]
{\normalfont\LARGE\bfseries}{\chaptertitlename\ \thechapter}{20pt}{\Huge}
\titlespacing{\chapter}{0pt}{0pt}{0pt}
%\titlespacing{\subsection}{0pt}{\parskip}{0pt}
%\titlespacing{\section}{0pt}{\parskip}{0pt}

% Gjem tekst
\newcommand\gj[1]{\begin{comment} #1 \end{comment}}

%Footnote:
\usepackage[bottom, hang, flushmargin]{footmisc}
\usepackage{perpage} 
\MakePerPage{footnote}
\addtolength{\footnotesep}{2mm}
\renewcommand{\thefootnote}{\arabic{footnote}}
\renewcommand\footnoterule{\rule{\linewidth}{0.4pt}}

%asin, atan, acos
\DeclareMathOperator{\atan}{atan}
\DeclareMathOperator{\acos}{acos}
\DeclareMathOperator{\asin}{asin}

%Tabell
\addto\captionsenglish{\renewcommand{\tablename}{Tabell}}

% Tikz
\usetikzlibrary{calc}
\usepackage{tkz-euclide}
\tikzset{
	font={\fontsize{11pt}{12}\selectfont}}
\usepackage{pgfplots}
\usetikzlibrary{matrix}

%Nummererte ligninger
%\newcommand\nreq[1]{\begin{equation*} #1 \end{equation*}}
\newcommand\nreq[1]{\begin{equation} #1 \end{equation}}

\newcommand{\scr}[1]{/home/sindre/R/scr/#1}
\newcommand{\asym}[1]{../asymptote/#1}
%Toc for seksjoner
\newcommand\tsec[1]{\phantomsection \addcontentsline{toc}{section}{#1}
	\section*{#1}}
%\newcommand\tssec[1]{\subsection*{#1}\addcontentsline{toc}{subsection}{#1}}
\newcommand\tssec[1]{\subsection{#1}}

% GeoGebra
\newcommand\cm[1]{{\large \tt #1} \gvs\\}
\newcommand\cmc[1]{{\large \tt #1} {\large (CAS)} \gvs\\}
\newcommand\cmk[1]{{\large \tt #1} {\large (Inntastingsfelt)} \gvs\\}
\newcommand\gvs{\vspace{\parskip}}

% Brok
\newcommand\br{\\[5 pt]}

% Opg
\newcommand{\opgt}{\phantomsection \addcontentsline{toc}{section}{Oppgaver} \section*{Oppgaver for kapittel \thechapter}}
\newcounter{opg}
\numberwithin{opg}{section}
\newcommand{\op}{\refstepcounter{opg} \textbf{\theopg} \vspace{2 pt} \\}
\newcommand{\opl}[1]{\vsk \refstepcounter{opg} \textbf{\theopg} \vspace{2 pt} \label{#1} \\}
\newcommand{\ekspop}{\vsk\textbf{Gruble \thechapter}\vspace{2 pt} \\}
\newcommand{\nes}{\stepcounter{section}
	\setcounter{opg}{0}}
\newcommand{\ness}{\stepcounter{subsection}
	\setcounter{opg}{0}}
\newcommand{\opr}[1]{\textbf{\ref{#1}}}
\newcommand{\se}[1]{Se eksempel s. {\pageref{#1}}}
\newcommand{\sel}{Se løsningsforslag.}

% Kolonner
%\usepackage{multicol}

%Vedlegg
\newcounter{vedl}
\newcounter{vedleq}
\renewcommand\thevedl{\Alph{vedl}}	
\newcommand{\vedlegg}[1]{\refstepcounter{vedl}\section*{Vedlegg \thevedl: #1}  \setcounter{vedleq}{0}}
\newcommand{\nreqvd}{\refstepcounter{vedleq}\tag{\thevedl \thevedleq}}

%page number
\usepackage{fancyhdr}
\pagestyle{fancy}
\fancyhf{}
\renewcommand{\headrule}{}
\fancyhead[RO, LE]{\thepage}

%more spaces
\newcommand{\regv}{\vspace{5pt}}

%equation
\newcommand{\y}[1]{$ {#1} $}

% index
\usepackage{imakeidx}
\makeindex[title=Indeks]

%fotnoter i tekstbokser med arabiske nummer
\renewcommand{\thempfootnote}{\arabic{mpfootnote}}

\usepackage[]{hyperref}

\newcommand{\eqlen}{
	%\setlength\abovedisplayskip{8pt plus 1mm minus 1mm}
	%	\setlength\belowdisplayskip{8pt plus 3pt minus 6pt}
	%	\setlength\abovedisplayshortskip{0pt plus 3pt}
	%	\setlength\belowdisplayshortskip{8pt plus 3.5pt minus 3pt}
	%\setlength\abovedisplayskip{8pt plus 0mm minus 0mm}
	%\setlength\belowdisplayskip{8pt plus 0pt minus 0pt}
	%\setlength\abovedisplayshortskip{0pt plus 0pt minus 0pt}
	%\setlength\belowdisplayshortskip{8pt plus 0pt minus 0pt}
	\allowdisplaybreaks
}

\usepackage{datetime2}

\usepackage{xr}
\externaldocument{/home/sindre/R/bokR2_PDF}
\setlength{\parskip}{11pt}
\begin{document}
\footnotesize
\opr{veiformel2}
\textbf{a)} \algv{
\int y''\,dt &= \int -g\,dt \\
\int y' \,dt&= -\int (gt + C)\,dt \\
y &= -\frac{1}{2}gt^2+Ct+D
}
\textbf{b)} \algv{y(0)&=-\frac{1}{2}g\cdot0+C\cdot 0 + D \\
0 &= D \\
&\\
y'(0)&= -g\cdot0+C \\
v_0 &= C}
Den spesifikke løsningen er derfor $ y=v_0-\frac{1}{2}gt^2 $, ofte nevnt som én av \textit{veiformlene} i fysikk.

\opr{kjernederye}\\
Av kjerneregelen ved derivasjon har vi at:
\alg{
\left(e^{F(x)}\right)'= e^{F(x)}f(x)
}
Av produktregelen ved derivasjon har vi da at:
\alg{
\left(y(x) e^{F(x)}\right)' &= y'(x)e^{F(x)}+y(x)e^{F(x)}f(x) 
}
Altså har vi vist det vi skulle.

\opr{FODE}
\textbf{a)} Vi har at:
\alg{
	\int 4\,dx &= 4x+ C
}
Den integrerende faktoren er derfor $ e^{4x}=$:
\alg{
(y'+4y)e^{4x}&=8e^{4x} \\
\left(ye^{4x}\right)' &= 8e^{4x} \\
\int \left(ye^{4x}\right)'\,dx &= \int  8e^{4x} \,dx \\
y  e^{4x} &= 2e^{4x} + C \\
y &= 2+Ce^{-4x}
}
\vsk
\textbf{b)} Vi har at:
\alg{
	\int \frac{1}{x}\,dx &= \ln x+ C
}
Den integrerende faktoren er derfor $ e^{\ln x}=x $:
\alg{
\left(y' + \frac{1}{x}y \right)x&= x\cos x \\
(yx)' &= x\cos x \\
\int (yx)'\,dx &= \int x\cos x\,dx \\
yx &= x\sin x+\int \sin x \,dx \\
yx &= x\sin x-\cos x + C \\
y &=\sin x + x^{-1}(\cos x+C)
}
\vsk
\textbf{c)}
Vi har at:
\alg{
\int \frac{3}{x}\,dx &= 3\ln x+ C
}
Den integrerende faktoren er derfor $ e^{3\ln x}=x^3 $:
\alg{
y'x^3 + \frac{3}{x}yx^3 &= x^3(15x +4) \\
\left(yx^3\right)' &= (15x^4+4x^3) \\\
\int \left(yx^3\right)' \,dx &=\int (15x^4+4x^3)\, dx \\
yx^3 &= 3x^5+x^3+C \\
y &= 3x^2+x+Cx^{-3}
}
\textbf{d)} Vi har at:
\alg{
	\int 3x^2\,dx &= x^3+ C
}
Den integrerende faktoren er derfor $ e^{x^3} $:
\alg{
(y'+3x^2 y)e^{x^3} &=(1+3x^2)e^x e^{x^3} \\
\left(ye^{x^3}\right)' &= (1+3x^2)e^{x^3+x} \\
\int \left(ye^{x^3}\right)' \,dx &= \int (1+3x^2)e^{x^3+x}\,dx\\
}
Vi setter $ {u=x^3+x} $, og får at:
\alg{
ye^{x^3} &= \int u' e^u\,dx \\
	&= \int e^u \,du \\
	&= e^u \\
ye^{x^3} &= e^{x^3+x} + C \\
y &= e^x + Ce^{-x^3}	
}

\opr{folketall}\\
\textbf{a)} Hvis vi lar $ y $ betegne folketallet, får vi ligningen:
\[ y'=ky \]
hvor $ k>0 $ siden $ y $ hele tiden er voksende.

\textbf{b)} Den generelle løsningen av ligningen i a) er $ y=Ce^{kt} $. Videre har vi at:
\alg{
y(0)&= Ce^{k\cdot0}\\
100 &=C
}
og at:
\alg{
y(1)&=100e^{k\cdot1} \\
\ln 101 &= \ln \left(100e^{k}\right) \\
\ln 101 &= \ln 100+\ln e^k\\
\ln\left(\frac{101}{100}\right)&= k 
}
Altså kan vi skrive:
\alg{
y &= 100e^{\ln\left(\frac{101}{100}\right)t} \\
&= 100\cdot1.01^t
}

\opr{newtavkj}
\textbf{a)}
\alg{
T'+kT&=kT_a \\
T'e^{kt}+kTe^{kt}&= kT_ae^{kt} \\
\left(Te^{kt}\right)'&= kT_ae^{kt} \\
\int \left(Te^{kt}\right)'\,dt&= \int kT_ae^{kt}\,dt \\
Te^{kt} &= T_ae^{kt} + C \\
T&= T_a+Ce^{-kt}
}
\textbf{b)} Siden $ T(0)=95 $ og $ T_a=15 $, har vi at:
\alg{
T(0)&= 15+Ce^{-k\cdot0} \\
95 &= 15+C \\
80 &= C 
}
Altså får vi at:
\[ T = 15+80e^{-\frac{\ln 2}{5}t} \]

\textbf{c)} \alg{
T(15) &= 15+80e^{-\frac{\ln 2}{5}\cdot15} \\
&= 15+80e^{-3\ln 2}\\
&= 15+80\cdot 2^{-3}\\
&= 25
}
\textbf{d)} $\lim\limits_{t\to\infty} e^{-kt}= 0 $, derfor vil temperaturen til gjenstanden gå må mot romtemperaturen.

\opr{sepode} \\
\textbf{a)} \se{arg1} og opg. b)

\textbf{b)} \algv{2\sqrt{x}y'&=\cos^2 y \\
\frac{y'}{\cos^2 y}&= \frac{1}{2\sqrt{x}} \\
\int \frac{y'}{\cos^2 y}\,dx&=\int \frac{1}{2\sqrt{x}}\,dx \\
\int \frac{1}{\cos^2 y}\,dy &=\frac{1}{2}\int x^{-\frac{1}{2}} \\
\tan y &= \sqrt{x}+ C \\
y &= \atan(\sqrt{x}+C)
}
\alg{
y(4)&= \atan(\sqrt{4}+C) \\
\frac{\pi}{4}&=\atan(2+C)
}
Siden $ \atan 1 = \frac{\pi}{4} $ må $ C=-1 $, altså har vi at:
\[ y =\atan(\sqrt{x}+1) \]

\opr{fjormassopg}\\
Et fjør-masse system uten demping er beskrevet av ligningen
\[ my''+ky =0 \]
hvor $ k>0 $. Den karakteristiske ligningen blir da:
\alg{
mr^2+k&=0 \\
r^2 &= -\frac{k}{m} \\
r &= \pm \text{i}\sqrt{\frac{k}{m}}
}
For to konstanter $ C $ og $ D $ er derfor den generelle løsningen gitt som:
\[y= C\cos\left(\sqrt{\frac{k}{m}}t\right)+D \sin\left(\sqrt{\frac{k}{m}}t\right)\]
Og dermed har vi vist det vi skulle.

\opr{fjormassmeddemp}\\
Et fjør-masse system med demping er beskrevet av ligningen
\[ my''+by'+ky=0 \]
Som vi kan omskrive til
\[ y'' +\frac{b}{m}y'+\frac{k}{m}y = 0 \]
Setter vi  $ \frac{b}{m}=2\alpha $ og $ \sqrt{\frac{k}{m}}=\omega $ får vi
\[ y''+2\alpha y'+\omega^2y=0 \]
som var det vi skulle vise.

\textbf{b)} Den karakteistiske ligninen blir:
\[  r^2+2\alpha r+\omega^2=0 \]
Løser vi denne ved abc-formelen får vi:
\alg{
r&= \frac{-2\alpha\pm\sqrt{(2\alpha)^2-4\cdot1\cdot \omega^2}}{2\cdot1} \\
&= \frac{-2\alpha\pm2\sqrt{\alpha^2-\omega^2}}{2\cdot1} \\
&= -\alpha\pm \sqrt{\alpha^2-\omega^2}
}
som var det vi skulle vise.

\textbf{c)}\vs \begin{itemize}
	\item Når $ \alpha>\omega $ får den karakteristiske ligningen to reelle løsninger siden uttrykket i kvadratroten blir et tall større enn 0. Løsningen av differensialligningen er dermed gitt ved (\ref{tore}).
	\item Når $ \alpha=\omega $ får den karakteristiske ligningen én reell løsning siden uttrykket i kvadratroten blir 0. Løsningen av differensialligningen er dermed gitt ved (\ref{enre}).
	\item Når $ \alpha<\omega $ får den karakteristiske ligningen to komplekse løsninger siden uttrykket i kvadratroten blir et negativt tall forskjellig fra 0. Løsningen av differensialligningen er dermed gitt ved (\ref{kompleks}).	
\end{itemize}
\textbf{d)} Vi faktoriserer den karakteristiske ligningen for enklere å avsløre oppførselen til uttrykket:
\[  -\alpha\pm \sqrt{\alpha^2-\omega^2} = -\alpha\pm\alpha \sqrt{1-\frac{\omega^2}{\alpha^2}}\]

Siden $ m $, $ q $ og $ k $ alle er positive tall forskjellige fra null, må også $ \alpha $ og $ \omega $ være det. Av ligningen over ser vi at hvis $ {\alpha>\omega} $, blir løsningen av den karakteristiske ligningen lik $ -\alpha $ pluss/minus et tall som er mindre enn $ \alpha $. Altså må begge løsninger bli negative og $ y $ blir da synkende for alle $ t>0 $. Videre ser vi at hvis $ {\alpha=\omega} $, blir løsningen av den karakteristiske ligningen lik $ -\alpha $. $ y $ består da av et ledd som er synkende for alle $ t $ og et ledd som går mot 0 når $ {t\to\infty} $. Til slutt ser vi at når $ {\alpha<\omega }$, gir rotutrykket opphav til en kompleks løsning. $ y $ består da av et sinus og cosinusuttrykk, som begge er multiplisert med $ e^{-\alpha t} $. Også da vil altså $ y $ gå mot 0 når $ t\to\infty $.
\end{document}