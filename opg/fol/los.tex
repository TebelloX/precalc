\documentclass[english, 11 pt, class=article, crop=false]{standalone}
%\documentclass[english, 11 pt]{report}
\usepackage[T1]{fontenc}
\usepackage[utf8]{luainputenc}
\usepackage{babel}
\usepackage{xr-hyper}

\usepackage{geometry}
%\geometry{verbose,margin=1.5cm,bmargin=2.5cm,lmargin=4.5cm,rmargin=4.5cm,headheight=10cm,headsep=1cm,footskip=1cm}
\geometry{verbose,paperwidth=16.1 cm, paperheight=24 cm, inner=2.3cm, outer=1.8 cm, bmargin=2cm, tmargin=1.8cm}
\setlength{\parindent}{0bp}
\usepackage{amsmath}
\usepackage{amssymb}
\usepackage{esint}
\usepackage{import}
\usepackage[subpreambles=false]{standalone}
%\makeatletter
\usepackage{tocloft}
\usepackage{graphicx}
\usepackage{placeins}
\usepackage{calc}
\usepackage{cancel}
\usepackage{color}
\definecolor{shadecolor}{rgb}{0.105469, 0.613281, 1}
\usepackage{framed}
\usepackage{wrapfig}
\usepackage{bm}
\usepackage{ntheorem}
\usepackage{ragged2e}
\RaggedRight
\raggedbottom
%\flushbottom
\frenchspacing

\newcounter{lign}[chapter]
\newenvironment{lign}[1][]{\Large \refstepcounter{lign} \large
	\textbf{\thelign #1} \rmfamily}{\par\medskip}
\numberwithin{lign}{chapter}
\numberwithin{equation}{chapter}
\usepackage{xcolor}
\usepackage{icomma}
\usepackage{mathtools}
\usepackage{lmodern} % load a font with all the characters
\makeatother
\usepackage[many]{tcolorbox}

\newcommand{\parskiplength}{11pt}
%\setlength{\parskip}{\parskiplength}

\newcommand\eks[2][]{\begin{tcolorbox}[arc=0mm,enhanced jigsaw,breakable,colback=green!8,boxrule=0.3 mm] {\large \textbf{Eksempel #1} \vspace{5 pt}\newline} #2 \vspace{1pt} \end{tcolorbox}\vspace{-4pt}}
\newcommand\tit[2][]{\large \textbf{#2 - Eksempel #1} \\}
\newcommand\lig[1]{\begin{lign} \large \normalfont #1 \end{lign}}
\newcommand\hrds[1]{\hyperref[#1]{\textsl{delseksjon \ref*{#1}}}}
\newcommand\hr[2][]{\hyperref[#2]{\textsl{#1}}}
\newcommand\hrs[2][]{\hyperref[#2]{\textsl{\textsl{#1} \ref*{#2}}}}
\newcommand\hrv[1]{\hyperref[#1]{\textsl{\textsl{vedlegg} \ref*{#1}}}}
\newcommand\fref[2][]{\hyperref[#2]{\textsl{figur \ref*{#2}#1}}}
\newcommand\hrss[2][]{\hyperref[#2]{\textsl{#1}}}
\newcommand\ligg[1]{{\large \normalfont \textbf{#1}} \vspace{5 pt}\\}
%\newcommand\rg[2][]{\begin{tcolorbox}[colback=blue!15] \ligg{#1} #2  %\end{tcolorbox}}
\newcommand\rg[2][]{\begin{tcolorbox}[arc=0mm,colback=blue!5,enhanced jigsaw,breakable, boxrule=0.3 mm]{\large \normalfont \textbf{#1} \vspace{5 pt}\newline} #2 \vspace{1pt} \end{tcolorbox}\vspace{-4pt}}
\newcommand\rgg[2][]{\begin{tcolorbox}[colback=orange!55] #2 \vspace{1pt} \end{tcolorbox}\vspace{-9pt}}
\newcommand\alg[1]{\begin{align*} #1 \end{align*}}
\newcommand{\algv}[1]{\vspace{-9 pt} \begin{align*} #1 \end{align*}}
\newcommand{\algb}[1]{\vspace{-9 pt} \begin{align*} #1 \end{align*}}
\newcommand\vs{\vspace{-\parskiplength}}
\newcommand\vsb{\vspace{-13pt}}
\newcommand\vds{\vs\vs}
\newcommand\g[1]{\begin{center} \vspace{-11 pt} {\tt #1} \vspace{-11 pt} \end{center}}
\newcommand\gv[1]{\begin{center} \vspace{-22 pt} {\tt #1} \vspace{-11 pt} \end{center}}
\newcommand\vsk{\vspace{11 pt}}
%\addto\captionsenglish{\renewcommand{\contentsname}{Løsningsforslag tentamen R2 H2015}}

% Farger
%\pagecolor{yellow!3}
\colorlet{shadecolor}{blue!30} 

% Figur
\usepackage{float}
\usepackage{subfig}
\captionsetup[subfigure]{labelformat=empty}
\newcommand{\fig}[2]{\begin{figure}
		\centering
		\includegraphics[]{\asym{#1}}
		\caption{#2}
\end{figure}}
\newcommand{\net}[2]{{\color{blue}\href{#1}{#2}}}

\usepackage{esvect}
\usepackage[font=footnotesize,labelfont=sl]{caption}
\addto\captionsenglish{\renewcommand{\figurename}{Figur}}

\newcommand{\sss}[1]{\subsection*{#1}   \addcontentsline{toc}{subsection}{#1}}
\newcommand\sv{\vsk \textbf{Svar:} \vspace{4 pt}\\}
\newcommand{\bs}{\\[4 pt]}
\newcommand{\os}{\\[4 pt]}
%Tableofconents
%\setlength{\cftsubsecindent}{1 cm}
\renewcommand{\cfttoctitlefont}{\Large\bfseries}
\setlength{\cftaftertoctitleskip}{0 pt}
\setlength{\cftbeforetoctitleskip}{0 pt}
\renewcommand\cftchapfont{\footnotesize\bfseries}
\renewcommand\cftsecfont{\footnotesize}
\renewcommand\cftsubsecfont{\footnotesize}
\addto\captionsenglish{\renewcommand{\contentsname}{Innhold}}
\addto\captionsenglish{\renewcommand{\chaptername}{Kapittel}}
\newcommand\tocskip{3 pt}
\setlength{\cftbeforechapskip}{12 pt}
\setlength{\cftbeforesecskip}{\tocskip}
\setlength{\cftbeforesubsecskip}{\tocskip}

%Seksjoner
\usepackage{titlesec}
\titleformat{\chapter}[display]
{\normalfont\LARGE\bfseries}{\chaptertitlename\ \thechapter}{20pt}{\Huge}
\titlespacing{\chapter}{0pt}{0pt}{0pt}
%\titlespacing{\subsection}{0pt}{\parskip}{0pt}
%\titlespacing{\section}{0pt}{\parskip}{0pt}

% Gjem tekst
\newcommand\gj[1]{\begin{comment} #1 \end{comment}}

%Footnote:
\usepackage[bottom, hang, flushmargin]{footmisc}
\usepackage{perpage} 
\MakePerPage{footnote}
\addtolength{\footnotesep}{2mm}
\renewcommand{\thefootnote}{\arabic{footnote}}
\renewcommand\footnoterule{\rule{\linewidth}{0.4pt}}

%asin, atan, acos
\DeclareMathOperator{\atan}{atan}
\DeclareMathOperator{\acos}{acos}
\DeclareMathOperator{\asin}{asin}

%Tabell
\addto\captionsenglish{\renewcommand{\tablename}{Tabell}}

% Tikz
\usetikzlibrary{calc}
\usepackage{tkz-euclide}
\tikzset{
	font={\fontsize{11pt}{12}\selectfont}}
\usepackage{pgfplots}
\usetikzlibrary{matrix}

%Nummererte ligninger
%\newcommand\nreq[1]{\begin{equation*} #1 \end{equation*}}
\newcommand\nreq[1]{\begin{equation} #1 \end{equation}}

\newcommand{\scr}[1]{/home/sindre/R/scr/#1}
\newcommand{\asym}[1]{../asymptote/#1}
%Toc for seksjoner
\newcommand\tsec[1]{\phantomsection \addcontentsline{toc}{section}{#1}
	\section*{#1}}
%\newcommand\tssec[1]{\subsection*{#1}\addcontentsline{toc}{subsection}{#1}}
\newcommand\tssec[1]{\subsection{#1}}

% GeoGebra
\newcommand\cm[1]{{\large \tt #1} \gvs\\}
\newcommand\cmc[1]{{\large \tt #1} {\large (CAS)} \gvs\\}
\newcommand\cmk[1]{{\large \tt #1} {\large (Inntastingsfelt)} \gvs\\}
\newcommand\gvs{\vspace{\parskip}}

% Brok
\newcommand\br{\\[5 pt]}

% Opg
\newcommand{\opgt}{\phantomsection \addcontentsline{toc}{section}{Oppgaver} \section*{Oppgaver for kapittel \thechapter}}
\newcounter{opg}
\numberwithin{opg}{section}
\newcommand{\op}{\refstepcounter{opg} \textbf{\theopg} \vspace{2 pt} \\}
\newcommand{\opl}[1]{\vsk \refstepcounter{opg} \textbf{\theopg} \vspace{2 pt} \label{#1} \\}
\newcommand{\ekspop}{\vsk\textbf{Gruble \thechapter}\vspace{2 pt} \\}
\newcommand{\nes}{\stepcounter{section}
	\setcounter{opg}{0}}
\newcommand{\ness}{\stepcounter{subsection}
	\setcounter{opg}{0}}
\newcommand{\opr}[1]{\textbf{\ref{#1}}}
\newcommand{\se}[1]{Se eksempel s. {\pageref{#1}}}
\newcommand{\sel}{Se løsningsforslag.}

% Kolonner
%\usepackage{multicol}

%Vedlegg
\newcounter{vedl}
\newcounter{vedleq}
\renewcommand\thevedl{\Alph{vedl}}	
\newcommand{\vedlegg}[1]{\refstepcounter{vedl}\section*{Vedlegg \thevedl: #1}  \setcounter{vedleq}{0}}
\newcommand{\nreqvd}{\refstepcounter{vedleq}\tag{\thevedl \thevedleq}}

%page number
\usepackage{fancyhdr}
\pagestyle{fancy}
\fancyhf{}
\renewcommand{\headrule}{}
\fancyhead[RO, LE]{\thepage}

%more spaces
\newcommand{\regv}{\vspace{5pt}}

%equation
\newcommand{\y}[1]{$ {#1} $}

% index
\usepackage{imakeidx}
\makeindex[title=Indeks]

%fotnoter i tekstbokser med arabiske nummer
\renewcommand{\thempfootnote}{\arabic{mpfootnote}}

\usepackage[]{hyperref}

\newcommand{\eqlen}{
	%\setlength\abovedisplayskip{8pt plus 1mm minus 1mm}
	%	\setlength\belowdisplayskip{8pt plus 3pt minus 6pt}
	%	\setlength\abovedisplayshortskip{0pt plus 3pt}
	%	\setlength\belowdisplayshortskip{8pt plus 3.5pt minus 3pt}
	%\setlength\abovedisplayskip{8pt plus 0mm minus 0mm}
	%\setlength\belowdisplayskip{8pt plus 0pt minus 0pt}
	%\setlength\abovedisplayshortskip{0pt plus 0pt minus 0pt}
	%\setlength\belowdisplayshortskip{8pt plus 0pt minus 0pt}
	\allowdisplaybreaks
}

\usepackage{datetime2}

\usepackage{xr}
\externaldocument{/home/sindre/R/bokR2_PDF}
\setlength{\parskip}{11pt}
\begin{document}
\opr{eksar}\\
\textbf{a)}
Vi bruker den eksplisitte fomelen for en aritmetisk følge, og får:
\alg{
	a_4 &= a_1 + d(i-1) \\
	30 &= 3 +d(4-1) \\
	27 &= 3d \\
	9 &= d
}
\textbf{b)}
Vi observerer at:
\alg{
	a_5-a_3 &= a_1+d(5-1)-(a_1+d(3-1)) \\
	a_5-a_3&= 2d \\
	26-14 &= 2d \\
	6 &= d
}
Videre har vi at:\vs
\alg{
	a_3 &= a_1+2d \\
	14 &= a_1 +12 \\
	2 &= a_1
}

\opr{eksgeo}\\
\textbf{a)}
Vi har at:
\[ k=\frac{a_2}{a_1}= \frac{1}{3}\]
Dermed er det eksplisitte uttrykket gitt som:
\alg{
a_n &= \frac{1}{2}\cdot\left(\frac{1}{3}\right)^{i-1} \\
&= \frac{1}{2}\cdot\frac{1}{3^{i-1}} \\
&= \frac{1}{2}\cdot3^{1-i}
}
\textbf{b)}
Vi vet at:
\algv{a_1\cdot k^{4-1}&= a_4 \\
5 \cdot k^3 &= 40 \\
k^3 &= 8 \\
k &= 2
}
Altså får vi:
\[ a_n = 5\cdot2^{i-1} \]

\opr{sum10ar}\\
\textbf{a)} Vi observerer at rekka er en aritmetisk rekke med $ a_1= 7$ og $ d= 6$. For å finne summen trenger vi verdien til $ a_{10} $:
\alg{a_{10} &= 7+6(10-1) \\
&= 61}
Summen $ S_{10} $ blir da:
\alg{
S_{10} &= 10\cdot\frac{7+61}{2}\\
&= 340
}
\textbf{b)} Se \textsl{a}.

\opr{ar435} \\
Rekken er aritmetisk med \y{a_1=8} og \y{d=3}. Vi har at:
\alg{
n\frac{8+(8+3(n-1))}{2} &= 435 \\
3n^2 +13n-870 &= 0 
}
Vi bruker \textit{abc}-formelen og får at $ {n\in\lbrace15, -\frac{58}{3}\rbrace} $, hvorav $ {n=15} $ er eneste mulige svar. 

\opr{viseks3} \\
\vs
\algv{
3\cdot9\cdot27\cdot\ldots \cdot 3^n &= 3^1\cdot3^2\cdot3^3\ldots\cdot3^n \\
&= 3^{1+2+\ldots+ n} \\
&=3^{n\frac{1+n}{2}} \\
&= 3^{\frac{1}{2}n(n+1)}
}
Som er det vi skulle vise.

\opr{geon} \\
Rekken er geometrisk, med $ {a_1=3} $ og $ {k=4} $. For å finne summen må vi vite hvor mange ledd rekken består av:
\alg{
3 \cdot 4 ^{n-1}=768 \\
4^{n-1} &= 256 \\
4^{n-1} &= 4^4\\
n-1 &= 4 \\
n &= 5
}

\opr{geoa12} \\
\textbf{a)}
Summen $ S_n $ er gitt som:
\algv{
S_n &= 2\cdot\frac{1-3^k}{1-3} \\
&= 2\cdot\frac{1-3^k}{-2} \\
&= 3^k-1
}
\textbf{b)} \algv{S_3&=3^3-1 \\
&= 26}

\textbf{c)} \algv{
3^n-1 &= 728\\
3^n &= 729\\
3^n &= 3^6\\
n &= 6
}

\opr{geospar}\\
\textbf{a)} Når du har spart i 4 måneder betyr det at første innskudd har forrentet seg 4 ganger, andre beløp tre ganger osv. Forrentingen tilsvarer en økning med $ 1.02 $. Medregnet det ferske innskuddet blir regnestykket:
\[ 1000\cdot1.02^4+1000\cdot1.02^3+1000\cdot1.02^2+1000\cdot1.02^1+1000 \]
\textbf{b)} Av oppgave \textsl{a} innse vi at $ P(n) $ er summe av en geometrisk rekke med $ a_1 = 1000 $ og $ k=1.02 $:
\alg{
P(n)&= 1000\cdot\frac{1-1.02^n}{1-1.02} \\
&= -50000(1-1.02^n) \\
&= 50000(1.02^n-1)
}

\opr{1over4}\\
\textbf{a)} Dette er en uendelig geometrisk rekke med $ k=\frac{1}{4} $. Siden $ |k|<1 $ er rekka konvergent.

\textbf{b)} Siden rekka er uendelig geometrisk og konvergent, har rekka en endelig sum $ S_\infty $ gitt ved:
\alg{
S_\infty&=\frac{a_1}{1-k} \br
&= \frac{4}{\frac{3}{4}}\\
&= \frac{16}{3}
}

\begin{comment}
	\opr{stav}\\
Dette blir en uendelig geometrisk rekke på formen:
\[ 1+\frac{1}{10}+\frac{1}{100}+... \]
Siden $ k=\frac{1}{10} $ er $ |k|<1 $ og derfor er rekka konvergent. Summen $ S_\infty $ er da gitt som:
\alg{
S_\infty&=\frac{a_1}{1-k} \\
&= \frac{1}{1-\frac{1}{10}} \\
&= \frac{1}{\frac{9}{10}} \br
&= \frac{10}{9}
}
Altså blir lengden $ \frac{10}{9} $ meter.
\end{comment}
\opr{099er1} \\
\textbf{a)} $ {\frac{9}{10}+\frac{9}{10^2}+\frac{9}{10^3}+\ldots} $. Dette er en geometrisk rekke med $ {a_1 = \frac{9}{10}} $ og $ k=10^{-1} $.
\textbf{b)} Fordi $ {|k|<1 }$ er rekken konvergent. Den uendelige summen er derfor gitt som:
\algv{
S_\infty &= \frac{\frac{9}{10}}{1-\frac{1}{10}} \\
&= \frac{\frac{9}{10}}{\frac{9}{10}} \\
&= 1
}
Summen av rekken blir 1, altså er \y{0.999...=1} (!).

\opr{geokonv} \\
\textbf{a)} Vi observerer at ${ k=x-2} $. Skal rekka konvergere må altså $ {|x-2|<1} $. Skal dette være sant må vi ha at:
\alg{
-1 &< x-2 \\
1 &< x
}
og videre at:
\algv{
x-2 &< 1 \\
x < 3
}
Derfor må vi ha at $ {1<x<3} $.

\textbf{b)} \algv{
\frac{\frac{1}{3}}{1-(x-2)}&=\frac{2}{9} \br
\frac{1}{3(3-x)} &= \frac{2}{9} \br
\frac{2}{18-6x} &= \frac{2}{9} \br
18-6x &= 9 \\
x &= \frac{3}{2}
}
$  {x=\frac{3}{2}} $ ligger i konvergensområdet, og er derfor et gyldig svar.

\textbf{c)} \algv{
\frac{\frac{1}{3}}{1-(x-2)}&=\frac{1}{6} \br
\frac{1}{3(3-x)} &= \frac{1}{6} \br
3(3-x) &= 6 \\
x &= 1
}
Men $  {x=1} $ ligger ikke i konvergensområdet, og er derfor ikke et gyldig svar. $ S_n=\frac{1}{6} $ har derfor ingen løsning.

\opr{ind}\\
\textbf{a)} Vi sjekker påstanden for $ {n=1}$:
\alg{1& = \frac{1(1+1)}{2} \\
1&= 1}
Påstanden er sann for $ {n=1} $, vi går derfor videre til å sjekke påstanden for $ {n=k+1} $. Når vi antar at formelen stemmer fram til ledd, $ k $ får vi:
\algv{
1+2+3+\ldots+(k+1) &= \frac{(k+1)(k+1+1)}{2} \br
\frac{k(k+1)}{2}+k+1 &= \frac{(k+1)(k+2)}{2} \br
\frac{k(k+1)+2(k+1)}{2} &= \\
\frac{(k+1)(k+2)}{2} &= \frac{(k+1)(k+2)}{2}
}
Dermed har vi vist det vi skulle.

\textbf{b)} Vi sjekker påstanden for $ {n=1} $:
\alg{1 &= 2^n-1 \\
1 &= 1
}
Påstanden er sann for $ n=1 $, vi går derfor videre til å sjekke påstanden for $ n=k+1 $. Når vi antar at formelen stemmer fram til ledd $ k $, får vi:
\algv{
1+2 +2^2 +...+ 2^{k+1-1}&= 2^{k+1}-1 \\
2^k-1+ 2^k &= \\
2\cdot 2^k-1 &= \\
2^{k+1}-1 &= 2^{k+1}-1
}
Dermed har vi vist det vi skulle.

\textbf{c)} 
Vi sjekker påstanden for $ {n=1}$:
\alg{4 &= \frac{4}{3}(4^1-1) \\
4&= \frac{4}{3}\cdot3\\
4&=4}
Påstanden er sann for $ {n=1} $, vi går derfor videre til å sjekke påstanden for $ {n=k+1} $. Når vi antar at formelen stemmer fram til ledd, $ k $ får vi:
\alg{
4+4^2+4^3+\ldots+4^{k+1}&=  \frac{4}{3}(4^{k+1}-1) \\
\frac{4}{3}(4^k-1)+4^{k+1} &=  \\
\frac{4^{k+1}-1+3\cdot 4^{k+1}}{3} &= \\
\frac{4}{3}(4^{k+1}-1)&=\frac{4}{3}(4^{k+1}-1)
}
Dermed har vi vist det vi skulle.

\textbf{d)} Vi sjekker påstanden for $ n=1 $:
\alg{1 &= \frac{1(2\cdot1+1)(1+1)}{6} \\
&= \frac{6}{6} \\
1&= 1
}
Påstanden er sann for $ n=1 $, vi går derfor videre til å sjekke påstanden for $ n=k+1 $. Når vi antar at formelen stemmer fram til ledd $ k $ får vi:
\alg{
1^2 + 2^2 + 3^3...+ (k+1)^2 &= \frac{(k+1)(2(k+1)+1)((k+1)+1)}{6} \br
\frac{k(2k+1)(k+1)}{6} + (k+1)^2 &= \frac{(k+1)(2k+3)(k+2)}{6} \br
\frac{k(2k+1)(k+1)+6(k+1)^2}{6} &= \br
\frac{(k+1)(k(2k+1)+6(k+1)}{6}&= \br
\frac{(k+1)(k(2k+1)+2k+4k+6)}{6} &= \br
\frac{(k+1)(k(2k+3)+4k+6)}{6} &= \br
\frac{(k+1)(k(2k+3)+2(2k+3)}{6} &= \br
\frac{(k+1)(k+2)(2k+3)}{6} &= \frac{(k+1)(2k+3)(k+2)}{6} 
}
Og dermed har vi vist det vi skulle. 

\textsl{Merk}: Faktorisering er en treningsak, men observer hvordan vi i overgangen mellom linje 5 og 6 framkalte leddet $ 2k+3 $. Hvis man ikke kommer i mål med ren faktorisering, kan man selvfølgelig etter linje 4 vise at $ k(2k+1)+6(k+1)=(2k+3)(k+2) $ ved å skrive ut uttrykkene på begge sider.

\opr{div3} \\
Vi sjekker påstanden for $ {n=1}$:
\alg{1(1^2+2) &= 1\cdot3
}
Påstanden er sann for $ {n=1} $, vi går derfor videre til å sjekke påstanden for $ {n=k+1} $. Når vi antar at formelen stemmer for $ n=k $, får vi:
\alg{
(k+1)((k+1)^2+2)&= (k+1)(k^2+2k+3) \\
&= (k+1)(k(k+2)+3)
}
Antakelsen vår sier at $ k(k+2) $ er delelig med 3, noe tallet 3 også er. Faktoren $ (k(k+2)+3) $ er derfor delelig med 3, mens $ (k+1) $ er et heltall. Uttrykket i ligningen over er derfor delelig med 3.
 
\opr{factorials} \\
\textbf{a)} Vi sjekker påstanden for $ {n=1}$:
\alg{
\frac{(2\cdot 1)!}{(2\cdot1-1)!} &= 2^1\cdot1!\br
\frac{1\cdot2}{1} &=  2 \\
2 &= 2
}
Påstanden er sann for $ {n=1} $, vi går derfor videre til å sjekke påstanden for $ {n=k+1} $. Når vi antar at formelen stemmer fram til ledd $ k $, får vi:
\alg{
\frac{1\cdot2}{1}\cdot\frac{1\cdot2\cdot3\cdot4}{1\cdot2\cdot3}\cdot
\ldots\cdot \frac{(2(k+1))!}{(2(k+1)-1)!}&=2^{k+1} (k+1)! \br
2^k k! \frac{(2(k+1))!}{(2k+1)!} &= \br
2^k k! \frac{(2k+1)!(2k+2)}{(2k+1)!} &= \br
2^{k+1}k!(k+1) &= \br
2^{k+1}(k+1)! &= 2^{k+1}(k+1)!
}
Dermed har vi vist det vi skulle.

\textbf{b)} Venstresiden kan enklere skrives som:
\[ 2\cdot4\cdot6\cdot\ldots\cdot2(k+1) \]
For $ {n=1} $:
\algv{
	2 &= 2^1\cdot1!\\
	2 &= 2
}
For $ {n=k+1} $:
\algv{
	2\cdot4\cdot6\cdot\ldots\cdot2(k+1)&=2^{k+1} (k+1)! \br
	2^k k!\cdot2(k+1) &=  \\
	2^{k+1}(k+1)!&= 2^{k+1} (k+1)!
}
\textbf{Gruble 1}\\
\textbf{a)} Summen av de $ n $ første oddetallene tilsvarer $ n^2 $ (se f. eks \ref{parodd}b), derfor kan vi skrive kvadratene som summer av oddetall.

\textbf{b)} Vi får $ n $ enere, $ {n-1} $ treere, $ {n-2} $ femmere og så videre. Den isolerte $ n $-en på høyresiden representerer de $ n $ enerene, mens summen
representerer bidragene fra alle de andre oddetallene (skriv opp hvis du syns det er vanskelig å se). 

\textbf{c)} \vs
\alg{
\sum\limits_{i=1}^n i^2 &= n+\sum\limits_{i=1}^n (n-i)(2i+1) \\
\sum\limits_{i=1}^n i^2 &= n+\sum\limits_{i=1}^n (2in+n-2i^2-i ) \\
\sum\limits_{i=1}^n i^2+\sum\limits_{i=1}^n 2i^2 &= n+\sum\limits_{i=1}^n ((2n-1)i+n ) \\
\sum\limits_{i=1}^n 3i^2 &= n+n^2+(2n-1)\frac{n(n+1)}{2} \\
\sum\limits_{i=1}^n i^2 &= \frac{2n(1+n)+(2n-1)n(n+1)}{6} \\
&= \frac{(2n+(2n-1)n)(n+1)}{6} \\
&= \frac{n(2+(2n-1))(n+1)}{6} \\
&= \frac{n(2n+1)(n+1)}{6}
}


\end{document}