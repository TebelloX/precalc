%\documentclass[english, 11 pt, class=article, crop=false]{standalone}
%%\documentclass[english, 11 pt]{report}
\usepackage[T1]{fontenc}
\usepackage[utf8]{luainputenc}
\usepackage{babel}
\usepackage[hidelinks, bookmarks]{hyperref}
\usepackage{geometry}
\geometry{verbose,tmargin=1cm,bmargin=3cm,lmargin=4cm,rmargin=4cm,headheight=3cm,headsep=1cm,footskip=1cm}
\setlength{\parindent}{0bp}
\usepackage{amsmath}
\usepackage{amssymb}
\usepackage{esint}
\usepackage{import}
\usepackage[subpreambles=false]{standalone}
%\makeatletter
\addto\captionsenglish{\renewcommand{\chaptername}{Kapittel}}
\makeatother
\usepackage{tocloft}
\addto\captionsenglish{\renewcommand{\contentsname}{Innhold}}
\usepackage{graphicx}
\usepackage{placeins}
\raggedbottom
\usepackage{calc}
\usepackage{cancel}
\makeatletter
\usepackage{color}
\definecolor{shadecolor}{rgb}{0.105469, 0.613281, 1}
\usepackage{framed}
\usepackage{wrapfig}
\usepackage{bm}
\usepackage{ntheorem}

\usepackage{ragged2e}
\RaggedRight
\raggedbottom
\frenchspacing

\newcounter{lign}[section]
\newenvironment{lign}[1][]{\Large \refstepcounter{lign} \large
	\textbf{\thelign #1} \rmfamily}{\par\medskip}
\numberwithin{lign}{section}
\numberwithin{equation}{section}
\usepackage{xcolor}
\usepackage{icomma}
\usepackage{mathtools}
\usepackage{lmodern} % load a font with all the characters
\usepackage{xr-hyper}
\makeatother
\usepackage[many]{tcolorbox}

%\setlength{\parskip}{\medskipamount}
\newcommand{\parskiplength}{11pt}
%\setlength{\parskip}{0 pt}
\newcommand\eks[2][]{\begin{tcolorbox}[enhanced jigsaw,boxrule=0.3 mm, arc=0mm,breakable,colback=green!30] {\large \textbf{Eksempel #1} \vspace{\parskiplength}\\} #2 \vspace{1pt} \end{tcolorbox}\vspace{1pt}}

\newcommand\fref[2][]{\hyperref[#2]{\textsl{Figur \ref*{#2}#1}}}
\newcommand{\hr}[2]{\hyperref[#2]{\color{blue}\textsl{#1}}}

\newcommand\rgg[2][]{\begin{tcolorbox}[boxrule=0.3 mm, arc=0mm,colback=orange!55] #2 \vspace{1pt} \end{tcolorbox}\vspace{-2pt}}
\newcommand\alg[1]{\begin{align*} #1 \end{align*}}
\newcommand\algv[1]{\vspace{-11 pt} \begin{align*} #1 \end{align*}}
\newcommand\vs{\vspace{-11 pt}}
\newcommand\g[1]{\begin{center} {\tt #1}  \end{center}}
\newcommand\gv[1]{\begin{center} \vspace{-22 pt} {\tt #1} \vspace{-11 pt} \end{center}}
%\addto\captionsenglish{\renewcommand{\contentsname}{Løsningsforslag tentamen R2 H2015}}

% Farger
\colorlet{shadecolor}{blue!30} 

% Figur
\usepackage{float}
\usepackage{subfig}
\captionsetup[subfigure]{labelformat=empty}
\usepackage{esvect}

\newcommand\sv{\textbf{Svar:} \vspace{5 pt} \\}

%Tableofconents
\renewcommand{\cfttoctitlefont}{\Large\bfseries}
\setlength{\cftsubsecindent}{2 cm}
\newcommand\tocskip{6 pt}
\setlength{\cftaftertoctitleskip}{30 pt}
\setlength{\cftbeforesecskip}{\tocskip}
%\setlength{\cftbeforesubsecskip}{\tocskip}

%Footnote:
\usepackage[bottom, hang, flushmargin]{footmisc}
\usepackage{perpage} 
\MakePerPage{footnote}
\addtolength{\footnotesep}{2mm}
\renewcommand{\thefootnote}{\arabic{footnote}}
\renewcommand\footnoterule{\rule{\linewidth}{0.4pt}}

%asin, atan, acos
\DeclareMathOperator{\atan}{atan}
\DeclareMathOperator{\acos}{acos}
\DeclareMathOperator{\asin}{asin}

%Tabell
\addto\captionsenglish{\renewcommand{\tablename}{Figur}}

% Figur
\usepackage[font=footnotesize,labelfont=sl]{caption}
\addto\captionsenglish{\renewcommand{\figurename}{Figur}}

% Figurer
\newcommand\scr[1]{/home/sindre/R/scr/#1}
\newcommand\asym[1]{/home/sindre/R/asymptote/#1}

%Toc for seksjoner
\newcommand\tsec[1]{\phantomsection\addcontentsline{toc}{section}{#1}
	\section*{#1}}
%\newcommand\tssec[1]{\subsection*{#1}\addcontentsline{toc}{subsection}{#1}}
\newcommand\tssec[1]{\subsection*{#1}}
% GeoGebra
\newcommand{\cms}[2]{{\tt #1( #2 )}}
\newcommand{\cm}[2]{{\large \tt #1( #2 )} \gvs \\}
\newcommand{\cmc}[2]{{\large \tt #1( #2 )} \large (CAS)  \gvs \\ \normalsize}
\newcommand{\cmk}[2]{{\large \tt #1( #2 )} \large (Inntastingsfelt)  \gvs \\ \normalsize}

\newcommand\gvs{\vspace{11 pt}}

\newcommand\vsk{\vspace{11 pt}}
\newcommand{\merk}{\vsk \textsl{Merk}: }
\newcommand{\fig}[1]{
\begin{figure}
	\centering
	\includegraphics[scale=0.5]{fig/#1}
\end{figure}
}
\newcommand{\figc}[1]{
		\centering
		\includegraphics[scale=0.5]{fig/#1}
}

% Opg
%\newcommand{\opgt}{\phantomsection \addcontentsline{toc}{section}{Oppgaver} \section*{Oppgaver for kapittel \thechapter}}
\newcounter{opg}
\numberwithin{opg}{section}

\newcommand{\opl}[1]{\vspace{15pt} \refstepcounter{opg} \textbf{\theopg} \vspace{2 pt} \label{#1} \\}

\\
\documentclass[english, 11 pt, class=article, crop=false]{standalone}
%\documentclass[english, 11 pt]{report}
\usepackage[T1]{fontenc}
\usepackage[utf8]{luainputenc}
\usepackage{babel}
\usepackage[hidelinks, bookmarks]{hyperref}
\usepackage{geometry}
\geometry{verbose,tmargin=1cm,bmargin=3cm,lmargin=4cm,rmargin=4cm,headheight=3cm,headsep=1cm,footskip=1cm}
\setlength{\parindent}{0bp}
\usepackage{amsmath}
\usepackage{amssymb}
\usepackage{esint}
\usepackage{import}
\usepackage[subpreambles=false]{standalone}
%\makeatletter
\addto\captionsenglish{\renewcommand{\chaptername}{Kapittel}}
\makeatother
\usepackage{tocloft}
\addto\captionsenglish{\renewcommand{\contentsname}{Innhold}}
\usepackage{graphicx}
\usepackage{placeins}
\raggedbottom
\usepackage{calc}
\usepackage{cancel}
\makeatletter
\usepackage{color}
\definecolor{shadecolor}{rgb}{0.105469, 0.613281, 1}
\usepackage{framed}
\usepackage{wrapfig}
\usepackage{bm}
\usepackage{ntheorem}

\usepackage{ragged2e}
\RaggedRight
\raggedbottom
\frenchspacing

\newcounter{lign}[section]
\newenvironment{lign}[1][]{\Large \refstepcounter{lign} \large
	\textbf{\thelign #1} \rmfamily}{\par\medskip}
\numberwithin{lign}{section}
\numberwithin{equation}{section}
\usepackage{xcolor}
\usepackage{icomma}
\usepackage{mathtools}
\usepackage{lmodern} % load a font with all the characters
\usepackage{xr-hyper}
\makeatother
\usepackage[many]{tcolorbox}

%\setlength{\parskip}{\medskipamount}
\newcommand{\parskiplength}{11pt}
%\setlength{\parskip}{0 pt}
\newcommand\eks[2][]{\begin{tcolorbox}[enhanced jigsaw,boxrule=0.3 mm, arc=0mm,breakable,colback=green!30] {\large \textbf{Eksempel #1} \vspace{\parskiplength}\\} #2 \vspace{1pt} \end{tcolorbox}\vspace{1pt}}

\newcommand\fref[2][]{\hyperref[#2]{\textsl{Figur \ref*{#2}#1}}}
\newcommand{\hr}[2]{\hyperref[#2]{\color{blue}\textsl{#1}}}

\newcommand\rgg[2][]{\begin{tcolorbox}[boxrule=0.3 mm, arc=0mm,colback=orange!55] #2 \vspace{1pt} \end{tcolorbox}\vspace{-2pt}}
\newcommand\alg[1]{\begin{align*} #1 \end{align*}}
\newcommand\algv[1]{\vspace{-11 pt} \begin{align*} #1 \end{align*}}
\newcommand\vs{\vspace{-11 pt}}
\newcommand\g[1]{\begin{center} {\tt #1}  \end{center}}
\newcommand\gv[1]{\begin{center} \vspace{-22 pt} {\tt #1} \vspace{-11 pt} \end{center}}
%\addto\captionsenglish{\renewcommand{\contentsname}{Løsningsforslag tentamen R2 H2015}}

% Farger
\colorlet{shadecolor}{blue!30} 

% Figur
\usepackage{float}
\usepackage{subfig}
\captionsetup[subfigure]{labelformat=empty}
\usepackage{esvect}

\newcommand\sv{\textbf{Svar:} \vspace{5 pt} \\}

%Tableofconents
\renewcommand{\cfttoctitlefont}{\Large\bfseries}
\setlength{\cftsubsecindent}{2 cm}
\newcommand\tocskip{6 pt}
\setlength{\cftaftertoctitleskip}{30 pt}
\setlength{\cftbeforesecskip}{\tocskip}
%\setlength{\cftbeforesubsecskip}{\tocskip}

%Footnote:
\usepackage[bottom, hang, flushmargin]{footmisc}
\usepackage{perpage} 
\MakePerPage{footnote}
\addtolength{\footnotesep}{2mm}
\renewcommand{\thefootnote}{\arabic{footnote}}
\renewcommand\footnoterule{\rule{\linewidth}{0.4pt}}

%asin, atan, acos
\DeclareMathOperator{\atan}{atan}
\DeclareMathOperator{\acos}{acos}
\DeclareMathOperator{\asin}{asin}

%Tabell
\addto\captionsenglish{\renewcommand{\tablename}{Figur}}

% Figur
\usepackage[font=footnotesize,labelfont=sl]{caption}
\addto\captionsenglish{\renewcommand{\figurename}{Figur}}

% Figurer
\newcommand\scr[1]{/home/sindre/R/scr/#1}
\newcommand\asym[1]{/home/sindre/R/asymptote/#1}

%Toc for seksjoner
\newcommand\tsec[1]{\phantomsection\addcontentsline{toc}{section}{#1}
	\section*{#1}}
%\newcommand\tssec[1]{\subsection*{#1}\addcontentsline{toc}{subsection}{#1}}
\newcommand\tssec[1]{\subsection*{#1}}
% GeoGebra
\newcommand{\cms}[2]{{\tt #1( #2 )}}
\newcommand{\cm}[2]{{\large \tt #1( #2 )} \gvs \\}
\newcommand{\cmc}[2]{{\large \tt #1( #2 )} \large (CAS)  \gvs \\ \normalsize}
\newcommand{\cmk}[2]{{\large \tt #1( #2 )} \large (Inntastingsfelt)  \gvs \\ \normalsize}

\newcommand\gvs{\vspace{11 pt}}

\newcommand\vsk{\vspace{11 pt}}
\newcommand{\merk}{\vsk \textsl{Merk}: }
\newcommand{\fig}[1]{
\begin{figure}
	\centering
	\includegraphics[scale=0.5]{fig/#1}
\end{figure}
}
\newcommand{\figc}[1]{
		\centering
		\includegraphics[scale=0.5]{fig/#1}
}

% Opg
%\newcommand{\opgt}{\phantomsection \addcontentsline{toc}{section}{Oppgaver} \section*{Oppgaver for kapittel \thechapter}}
\newcounter{opg}
\numberwithin{opg}{section}

\newcommand{\opl}[1]{\vspace{15pt} \refstepcounter{opg} \textbf{\theopg} \vspace{2 pt} \label{#1} \\}


\usepackage{xr}
\externaldocument{/home/sindre/R/bokR2_PDF}
\begin{document}
\opr{parlinjeo} \se{lirop}

\opr{krysslinj}
Vi bruker kravet for $ x $ og $ z $-koordinaten for å sette opp et ligningssystem:
\alg{
-3-2t &= -7-3s \tag{I} \label{krysI} \\
1-t &= s \tag{II} \label{krysII}
}
Av (\ref{krysII}) har vi et uttrykk for $ s $. Setter vi dette inn i (\ref{krysI}) får vi:
\alg{
-3-2t&=-7+3(1-t) \\
-3-2t&=-7+3-3t \\
t &=-1
}
Altså er $ t=-1 $ og $ s=2 $. For disse verdiene gir begge parameteriseringene punktet $ A=(-1, 1, 2) $.

\opr{finnparplan} \se{plpare}

\opr{finnparplan2} \se{plpare2}

\opr{finnplan} \se{plroe1}

\opr{finnplan2}\\
\textbf{a)} Av parameteriseringen ser vi at to retningsvektorer må være $ [2, 3, 0] $ og $ [0, 2, -1] $. 

\textbf{b)} En normalvektor for planet er gitt ved vektorproduktet av retningsvektorene:
\alg{
\left|\begin{matrix}
	\vec{e}_x & \vec{e}_y & \vec{e}_z \\
	2 & 3 & 0 \\
	0 & 2 & -1
\end{matrix}\right| &= \vec{e}_x \left|\begin{matrix}
3 & 0 \\
2 & -1 
\end{matrix}\right|-\vec{e}_y \left|\begin{matrix}
2 & 0 \\
0 & -1 
\end{matrix}\right|+\vec{e}_z \left|\begin{matrix}
2 & 3 \\
0 & 2 
\end{matrix}\right| \\
&= \vec{e}_x (-3-0)-\vec{e}_x(-2-0)+\vec{e}_z(4-0) \\
&= [-3, -2, 4]  
}
Av parameteriseringen ser vi at $ (-4, 2, 1) $ er et punkt i planet, derfor kan vi skrive:
\alg{
-3(x-(-4))+2(y-2)+4(z-1)&= 0\\
-3x-12-2y-4+4z-4&= 0\\
-3x -2y+4z -20&=0 
}

\opr{finnplan4} \\
Vi krever at:\vs
\alg{
	(-2, 1, 1)\cdot[3t, 5, t] &= 0 \\
	-6t+5+t &= 0 \\
	t &= 1
}
Altså er $ [3, 5, 1] $ en retningsvektor for planet. Ligningen til planet blir da:
\alg{
3(x-(-2))+5(y-1)+(z-1) &= 0 \\
3x+5y + z &= 0
}

\opr{kuleopg} \se{kulee}

\opr{kuleopg2}\\
\textbf{a)} Vi starter med å skrive de fullstendige kvadratene:
\alg{
x^2-6x &= (x^2-3)^2-3^2 \\
y^2+2y &= (y+1)^2-1^2 \\
z^2-10z &= (z-5)^2-5^2
}
Vi kan derfor skrive:
\alg{
(x^2-3)^2+(y+1)^2 +(z-5)^2 -14-9-1-25&= 0 \\
(x^2-3)^2+(y+1)^2 +(z-5)^2 &= 49 \\
(x^2-3)^2+(y+1)^2 +(z-5)^2 &= 7^2 
}
Altså har kula sentrom i $ S=(3, -1, 5) $ og radius $ r=7 $.

\textbf{b)} Vi setter koordinatene til $ A $ faktoriserte venstresiden av kuleligningen og får:
\alg{
(4-3)^2+(1+1)^2 +(6-5)^2 &= 6
}
Ligningen over representerer den kvadrerte avstanden mellom $ S $ og $ A $, siden $ 6<49 $ må $ A $ ligge inni kula.

For $ B $ får vi:
\[ (-6-3)^2+(-4+1)^2 +(1-5)^2 = 106 \]
Siden $ 106>49 $ ligger $ B $ utenfor kula.

\opr{toparllin}\\
\textbf{a)} Av ligningen ser vi at $ [3, -2, 1] $ er en normalvektor.

\textbf{b)} I ligningen for $ \beta $ ser vi at hvis $ x=y=0 $, så må også $ z=0 $. $ \beta $ inneholder derfor origo.

\textbf{c)} Avstanden $ h $ mellom $ \alpha $ og $ \beta $ må tilsvare avstande mellom $ \alpha $ og et punkt i $ \beta $. Vi bruker svaret fra b) og får:
\alg{
h &= \frac{|3\cdot0-2\cdot2+1\cdot0+12|}{|[3, -2, 1]|} \br
&= \frac{12}{\sqrt{9+4+1}} \br
&= \frac{12}{\sqrt{14}}
}

\opr{kuleopg3}\\
\textbf{a)} For å finne $ S $ skriver vi de fullstendige kvadratene:
\alg{
x^2-6x &= (x-3)^2-3^2 \\
y^2+4y &= (y+2)^2-2^2 \\
z^2 &= z^2
}
Kuleligningen kan vi derfor skrive som:
\alg{
(x-3)^2+(y+2)^2+z^2-23-3^2-2^2 = 0 \\
(x-3)^2+(y+2)^2+z^2 &= 36 \\
(x-3)^2+(y+2)^2+z^2 &= 6^2
}
Altså har vi $ S=(3, -2, 0) $.

\textbf{b)} Av ligningen til planet ser vi at $ [2, -1, -2] $ er en normalvektor for $ \alpha $. Dette må være en retningsvektor for linja som går gjennom $ A $ og $ S $, som dermed kan parameteriseres ved:
\[ l: \left\lbrace{
	\begin{array}{l}
	x=3+ 2t  \\
	y=-2-t   \\
	z=-2t
	\end{array}
}\right.  \]

\textbf{c)} \textit{Løsningsmetode 1:}
Linja og kuleflata skjærer der parameteriseringen til linja oppfyler kuleligningen:
\alg{
((3+2t)-3)^2+((-2-t)+2)^2+(-2t-0)^2 &= 36 \\
(2t)^2+(-t)^2+(-2t)^2 &= 36 \\
9t^2 &= 36 \\
t^2 &= 4 \\
t &= \pm 2
}
For $ t=-2 $ gir parameteriseringen punktet $ (-1, 0, 4) $ mens for $ t=2 $ får vi punktet $ (7, -4, -4) $.

\textit{Løsningsmetode 2:} Vi har funnet at $ [2, -1, -2] $ er en retningsvektor for linja gjennom $ A $ og $ S $, denne vektoren har lengde 3. Vi kan derfor lage oss en retningsvektor med lengde $ 1 $ ved å skrive $ \frac{1}{3}[2, -1, 2] $. Siden avstanden mellom $ S $ og de to punktene vi søker er lik radiusen 6, må de være gitt ved uttrykket
\alg{
S\pm 6\cdot\frac{1}{2}[2, -1, -2] &= S\pm2[2, -1, -2]
}
Regner man ut dette får man (selvølgelig) samme svar som for \textit{Løsningsmetode 1}.

\textbf{d)} \se{plpare3} eller bruk lignende resonnement som \textit{Løsningsmetode 2} i opg. c).

\textbf{e)} Radiusen $ R $ til sirkelen, radiusen $ r $ til kula og linjestykket $ AS $ utgjør en rettvinklet trekant. Av Pytagoras' setning har vi da at:
\alg{
R^2 &= r^2-|\vv{AS}|^2 \\
 &= 6^2-3^2 \\
&= 36-9 \\
&= 27 \\
R &= \pm\sqrt{27} 
}
$ R $ har altså lengden $ \sqrt{27} $.
\end{document}