\documentclass[english,hidelinks,pdftex, 11 pt, class=report,crop=false]{standalone}
%\documentclass[english, 11 pt]{report}
\usepackage[T1]{fontenc}
\usepackage[utf8]{luainputenc}
\usepackage{babel}
\usepackage{xr-hyper}

\usepackage{geometry}
%\geometry{verbose,margin=1.5cm,bmargin=2.5cm,lmargin=4.5cm,rmargin=4.5cm,headheight=10cm,headsep=1cm,footskip=1cm}
\geometry{verbose,paperwidth=16.1 cm, paperheight=24 cm, inner=2.3cm, outer=1.8 cm, bmargin=2cm, tmargin=1.8cm}
\setlength{\parindent}{0bp}
\usepackage{amsmath}
\usepackage{amssymb}
\usepackage{esint}
\usepackage{import}
\usepackage[subpreambles=false]{standalone}
%\makeatletter
\usepackage{tocloft}
\usepackage{graphicx}
\usepackage{placeins}
\usepackage{calc}
\usepackage{cancel}
\usepackage{color}
\definecolor{shadecolor}{rgb}{0.105469, 0.613281, 1}
\usepackage{framed}
\usepackage{wrapfig}
\usepackage{bm}
\usepackage{ntheorem}
\usepackage{ragged2e}
\RaggedRight
\raggedbottom
%\flushbottom
\frenchspacing

\newcounter{lign}[chapter]
\newenvironment{lign}[1][]{\Large \refstepcounter{lign} \large
	\textbf{\thelign #1} \rmfamily}{\par\medskip}
\numberwithin{lign}{chapter}
\numberwithin{equation}{chapter}
\usepackage{xcolor}
\usepackage{icomma}
\usepackage{mathtools}
\usepackage{lmodern} % load a font with all the characters
\makeatother
\usepackage[many]{tcolorbox}

\newcommand{\parskiplength}{11pt}
%\setlength{\parskip}{\parskiplength}

\newcommand\eks[2][]{\begin{tcolorbox}[arc=0mm,enhanced jigsaw,breakable,colback=green!8,boxrule=0.3 mm] {\large \textbf{Eksempel #1} \vspace{5 pt}\newline} #2 \vspace{1pt} \end{tcolorbox}\vspace{-4pt}}
\newcommand\tit[2][]{\large \textbf{#2 - Eksempel #1} \\}
\newcommand\lig[1]{\begin{lign} \large \normalfont #1 \end{lign}}
\newcommand\hrds[1]{\hyperref[#1]{\textsl{delseksjon \ref*{#1}}}}
\newcommand\hr[2][]{\hyperref[#2]{\textsl{#1}}}
\newcommand\hrs[2][]{\hyperref[#2]{\textsl{\textsl{#1} \ref*{#2}}}}
\newcommand\hrv[1]{\hyperref[#1]{\textsl{\textsl{vedlegg} \ref*{#1}}}}
\newcommand\fref[2][]{\hyperref[#2]{\textsl{figur \ref*{#2}#1}}}
\newcommand\hrss[2][]{\hyperref[#2]{\textsl{#1}}}
\newcommand\ligg[1]{{\large \normalfont \textbf{#1}} \vspace{5 pt}\\}
%\newcommand\rg[2][]{\begin{tcolorbox}[colback=blue!15] \ligg{#1} #2  %\end{tcolorbox}}
\newcommand\rg[2][]{\begin{tcolorbox}[arc=0mm,colback=blue!5,enhanced jigsaw,breakable, boxrule=0.3 mm]{\large \normalfont \textbf{#1} \vspace{5 pt}\newline} #2 \vspace{1pt} \end{tcolorbox}\vspace{-4pt}}
\newcommand\rgg[2][]{\begin{tcolorbox}[colback=orange!55] #2 \vspace{1pt} \end{tcolorbox}\vspace{-9pt}}
\newcommand\alg[1]{\begin{align*} #1 \end{align*}}
\newcommand{\algv}[1]{\vspace{-9 pt} \begin{align*} #1 \end{align*}}
\newcommand{\algb}[1]{\vspace{-9 pt} \begin{align*} #1 \end{align*}}
\newcommand\vs{\vspace{-\parskiplength}}
\newcommand\vsb{\vspace{-13pt}}
\newcommand\vds{\vs\vs}
\newcommand\g[1]{\begin{center} \vspace{-11 pt} {\tt #1} \vspace{-11 pt} \end{center}}
\newcommand\gv[1]{\begin{center} \vspace{-22 pt} {\tt #1} \vspace{-11 pt} \end{center}}
\newcommand\vsk{\vspace{11 pt}}
%\addto\captionsenglish{\renewcommand{\contentsname}{Løsningsforslag tentamen R2 H2015}}

% Farger
%\pagecolor{yellow!3}
\colorlet{shadecolor}{blue!30} 

% Figur
\usepackage{float}
\usepackage{subfig}
\captionsetup[subfigure]{labelformat=empty}
\newcommand{\fig}[2]{\begin{figure}
		\centering
		\includegraphics[]{\asym{#1}}
		\caption{#2}
\end{figure}}
\newcommand{\net}[2]{{\color{blue}\href{#1}{#2}}}

\usepackage{esvect}
\usepackage[font=footnotesize,labelfont=sl]{caption}
\addto\captionsenglish{\renewcommand{\figurename}{Figur}}

\newcommand{\sss}[1]{\subsection*{#1}   \addcontentsline{toc}{subsection}{#1}}
\newcommand\sv{\vsk \textbf{Svar:} \vspace{4 pt}\\}
\newcommand{\bs}{\\[4 pt]}
\newcommand{\os}{\\[4 pt]}
%Tableofconents
%\setlength{\cftsubsecindent}{1 cm}
\renewcommand{\cfttoctitlefont}{\Large\bfseries}
\setlength{\cftaftertoctitleskip}{0 pt}
\setlength{\cftbeforetoctitleskip}{0 pt}
\renewcommand\cftchapfont{\footnotesize\bfseries}
\renewcommand\cftsecfont{\footnotesize}
\renewcommand\cftsubsecfont{\footnotesize}
\addto\captionsenglish{\renewcommand{\contentsname}{Innhold}}
\addto\captionsenglish{\renewcommand{\chaptername}{Kapittel}}
\newcommand\tocskip{3 pt}
\setlength{\cftbeforechapskip}{12 pt}
\setlength{\cftbeforesecskip}{\tocskip}
\setlength{\cftbeforesubsecskip}{\tocskip}

%Seksjoner
\usepackage{titlesec}
\titleformat{\chapter}[display]
{\normalfont\LARGE\bfseries}{\chaptertitlename\ \thechapter}{20pt}{\Huge}
\titlespacing{\chapter}{0pt}{0pt}{0pt}
%\titlespacing{\subsection}{0pt}{\parskip}{0pt}
%\titlespacing{\section}{0pt}{\parskip}{0pt}

% Gjem tekst
\newcommand\gj[1]{\begin{comment} #1 \end{comment}}

%Footnote:
\usepackage[bottom, hang, flushmargin]{footmisc}
\usepackage{perpage} 
\MakePerPage{footnote}
\addtolength{\footnotesep}{2mm}
\renewcommand{\thefootnote}{\arabic{footnote}}
\renewcommand\footnoterule{\rule{\linewidth}{0.4pt}}

%asin, atan, acos
\DeclareMathOperator{\atan}{atan}
\DeclareMathOperator{\acos}{acos}
\DeclareMathOperator{\asin}{asin}

%Tabell
\addto\captionsenglish{\renewcommand{\tablename}{Tabell}}

% Tikz
\usetikzlibrary{calc}
\usepackage{tkz-euclide}
\tikzset{
	font={\fontsize{11pt}{12}\selectfont}}
\usepackage{pgfplots}
\usetikzlibrary{matrix}

%Nummererte ligninger
%\newcommand\nreq[1]{\begin{equation*} #1 \end{equation*}}
\newcommand\nreq[1]{\begin{equation} #1 \end{equation}}

\newcommand{\scr}[1]{/home/sindre/R/scr/#1}
\newcommand{\asym}[1]{../asymptote/#1}
%Toc for seksjoner
\newcommand\tsec[1]{\phantomsection \addcontentsline{toc}{section}{#1}
	\section*{#1}}
%\newcommand\tssec[1]{\subsection*{#1}\addcontentsline{toc}{subsection}{#1}}
\newcommand\tssec[1]{\subsection{#1}}

% GeoGebra
\newcommand\cm[1]{{\large \tt #1} \gvs\\}
\newcommand\cmc[1]{{\large \tt #1} {\large (CAS)} \gvs\\}
\newcommand\cmk[1]{{\large \tt #1} {\large (Inntastingsfelt)} \gvs\\}
\newcommand\gvs{\vspace{\parskip}}

% Brok
\newcommand\br{\\[5 pt]}

% Opg
\newcommand{\opgt}{\phantomsection \addcontentsline{toc}{section}{Oppgaver} \section*{Oppgaver for kapittel \thechapter}}
\newcounter{opg}
\numberwithin{opg}{section}
\newcommand{\op}{\refstepcounter{opg} \textbf{\theopg} \vspace{2 pt} \\}
\newcommand{\opl}[1]{\vsk \refstepcounter{opg} \textbf{\theopg} \vspace{2 pt} \label{#1} \\}
\newcommand{\ekspop}{\vsk\textbf{Gruble \thechapter}\vspace{2 pt} \\}
\newcommand{\nes}{\stepcounter{section}
	\setcounter{opg}{0}}
\newcommand{\ness}{\stepcounter{subsection}
	\setcounter{opg}{0}}
\newcommand{\opr}[1]{\textbf{\ref{#1}}}
\newcommand{\se}[1]{Se eksempel s. {\pageref{#1}}}
\newcommand{\sel}{Se løsningsforslag.}

% Kolonner
%\usepackage{multicol}

%Vedlegg
\newcounter{vedl}
\newcounter{vedleq}
\renewcommand\thevedl{\Alph{vedl}}	
\newcommand{\vedlegg}[1]{\refstepcounter{vedl}\section*{Vedlegg \thevedl: #1}  \setcounter{vedleq}{0}}
\newcommand{\nreqvd}{\refstepcounter{vedleq}\tag{\thevedl \thevedleq}}

%page number
\usepackage{fancyhdr}
\pagestyle{fancy}
\fancyhf{}
\renewcommand{\headrule}{}
\fancyhead[RO, LE]{\thepage}

%more spaces
\newcommand{\regv}{\vspace{5pt}}

%equation
\newcommand{\y}[1]{$ {#1} $}

% index
\usepackage{imakeidx}
\makeindex[title=Indeks]

%fotnoter i tekstbokser med arabiske nummer
\renewcommand{\thempfootnote}{\arabic{mpfootnote}}

\usepackage[]{hyperref}

\newcommand{\eqlen}{
	%\setlength\abovedisplayskip{8pt plus 1mm minus 1mm}
	%	\setlength\belowdisplayskip{8pt plus 3pt minus 6pt}
	%	\setlength\abovedisplayshortskip{0pt plus 3pt}
	%	\setlength\belowdisplayshortskip{8pt plus 3.5pt minus 3pt}
	%\setlength\abovedisplayskip{8pt plus 0mm minus 0mm}
	%\setlength\belowdisplayskip{8pt plus 0pt minus 0pt}
	%\setlength\abovedisplayshortskip{0pt plus 0pt minus 0pt}
	%\setlength\belowdisplayshortskip{8pt plus 0pt minus 0pt}
	\allowdisplaybreaks
}

\usepackage{datetime2}

\usepackage{xr}
\externaldocument{../../bokR2_PDF}
\begin{document}

\opr{leno} \se{arg1}

\opr{closest}
\algv{\vv{AB} &= [3-1 , 2-(-1) , 1-(-2)]\\
&= [2, 3, 3] \\
\left|\vv{AB}\right|&= \sqrt{20}\\
&\\
\vv{AC} &= [0-1 , 5-(-1) , 6-(-2)]\\
&= [-1, 6, 8] \\
\left|\vv{AC}\right|&= \sqrt{101}
}
Siden $ {\sqrt{101}>\sqrt{20}} $ er $ B $ nærmest $ A $.\vsk

\opr{lenfakt}\\
\textbf{a)}
\algv{
	|\vec{u}|&=\sqrt{(ad)^2 + (bd)^2 + (bd)^2}\\
	&= \sqrt{a^2 d^2 + b^2 d^2 + c^2 d^2}\\
	&= \sqrt{d^2(a^2+b^2+c^2)}\\
	&= d\sqrt{a^2 + b^2 + c^2}
}
\textbf{b)} Som i opg. a) kan vi også her skrive
\[ |\vec{u}|=\sqrt{d^2(a^2+b^2+c^2)}\]
men siden $ d^2 $ er et positivt tall, mens $ d $ er negativ, har vi at:
\[ d\neq \sqrt{d^2} \]
istedenfor er:
\[ |d|= \sqrt{d^2} \]
derfor kan vi skrive:
\[ |\vec{u}|=|d|\sqrt{a^2 + b^2 + c^2} \]
\opr{skalfaktor}
\algv{
\vec{u}\cdot\vec{v} &= [ad, bd, cd]\cdot[eh, fh, gh]\\
&= adeh+bdfh+cdgh \\
&= dh(ae+ bf+ cg)
}

\opr{skalproo}\\
\textbf{a)} \se{arg1}\\
\textbf{b)} \se{arg1}\\
\textbf{c)}
\algv{
\vec{a}\cdot\vec{b}&= \left[\dfrac{1}{5}, \dfrac{3}{5}, \dfrac{1}{5}\right]\cdot\vec{b}=[512, -128, 64] \br
&= \frac{1}{5}[1, 3, -1]\cdot 64[8, -2, 1]\br
&= \frac{64}{5}(8-6-1) \br
&= \frac{64}{5}
}

\opr{skalpro2o} \\
\textbf{a)} \se{arg1}
\textbf{b)} \se{arg1}

\opr{finntheta} 
Finn vinkelen mellom $ \vec{a} $ og $ \vec{b} $ når:

\textbf{a)} $ \vec{a}=[ 5 ,-5,  2]$ og $ \vec{b}=[ 3 ,-4 , 5] $
\algv{
|\vec{a}| &= \sqrt{5^2 +(-5)^2 +2^2} \\
&= \sqrt{54} \\
&= \sqrt{9\cdot6} \\
&= 3\sqrt 6\\
& \\
|\vec{b}| &= \sqrt{3^2+(-4)^2 + 5^2} \\
&= \sqrt{50}\\
&= \sqrt{25\cdot2}\\
&= 5\sqrt{2}
}
\algv{
	\vec{a}\cdot\vec{b} &= [ 5 ,-5,  2]\cdot\vec{b}=[ 3 ,-4 , 5] \\
	&=15+20+10\\
&= 45 }
\algv{\cos \theta &= \frac{\vec{a}\cdot\vec{b}}{|\vec{a}||\vec{b}|}\br
&= \frac{45}{3\sqrt{6}\cdot5\sqrt{2}} \br
&= \frac{3}{2\sqrt{3}}\br
&= \frac{3\sqrt{3}}{2\sqrt{3}\sqrt{3}} \br
&= \frac{\sqrt{3}}{2}
}
Dette betyr at $ \theta=30^\circ $.

\opr{forkort} \\
\textbf{a)} \se{}

\textbf{b)} 
\algv{
(\vec{a}+ \vec{b}+\vec{c})^2 &= (\vec{a}+ \vec{b}+\vec{c})\cdot(\vec{a}+ \vec{b}+\vec{c}) \\
&= \vec{a}^{\,2}+\vec{a}\cdot\vec{b}+\vec{a}\cdot\vec{c}+\vec{b}\cdot\vec{a}+\vec{b}^{\,2}+\vec{b}\cdot\vec{c}+\vec{c}\cdot\vec{a}+\vec{c}\cdot\vec{b}+\vec{c}^{\,2} \\
&= 1^2+0+ \vec{a}\cdot\vec{c}+0+2^2+0+\vec{c}\cdot\vec{a}+0+5^2 \\
&= 2(15+\vec{a}\cdot\vec{c})
}

\opr{orto} \\
\textbf{a)} \se{arg1}\\
\textbf{b)} \se{arg1} \\
\textbf{b)} \se{arg1}

\opr{torto}\\
\textbf{a)} \se{arg1}

\textbf{b)} Vi krever at:
\alg{\vec{u}\cdot\vec{v} &= 0 \\ 
[-5, -1, 6] \cdot[t, t^2, 1]&= 0 \\
-5t -t^2 +6 &= 0
}
Siden $ (-2)\cdot(-3) = 6 $ og $ -2+(-3)=-5 $ kan vi skrive at:
\[ (t-2)(t-3)=0 \]
Kravet er dermed oppfylt hvis $ t\in\lbrace2, 3\rbrace $. 

\opr{sjekkpar}
\textbf{a)} Vi regner fort ut at forholdet mellom både førstekomponentene og andrekomponenten er 2, men at forholdet mellom tredjekomponentene er $ -\frac{1}{2} $. Vektorene er derfor ikke parallelle.

\textbf{b)} Vi observerer at:
\[ \vec{b}=\frac{3}{7}[-3, 5, 2] \]
dermed er $ \vec{b} $ et multiplum av $ \vec{a} $ og da er $ \vec{a}||\vec{b} $.

\opr{finntpar}\\
\textbf{a)} Vi bruker forholdet mellom første- og tredjekomponententene for å sette opp en ligning for $ t $:
\alg{-\frac{t+3}{3}&=-\frac{16}{8} \br
t+3 &= 6 \\
t &= 3
}
Forholdet mellom andrekomponentene blir da:
\alg{
\frac{1-3}{1}&= -2
}
Forholdet er $ -2 $ for alle komponentene når $ t=3 $ og da er $ \vec{a}||\vec{b} $.

\textbf{b)} Også her bruker vi første- og tredjekomponententene for å sette opp en ligning for $ t $, fordi vi da får isolert det kvadratiske leddet:
\alg{
	-\frac{t^2+2}{3}&= -\frac{(5t^2+3)}{8} \\
	8t^2 + 16 &= 15 t^2 + 9 \\
	7 x^2 &= 7 \\
	x &= \pm 1
}
Når $ t=1 $ er forholdet mellom både førstkomponentene og tredjekomponentene lik
\[ -\frac{1^2+2}{3}=-1 \]
Og forholdet mellom andrekomponentene er:
\[ \frac{1}{1}=1 \]
For $ t=1 $ er altså $ \vec{a} $ og $ \vec{b} $ ikke parallelle. Vi ser derimot fort at forholdet mellom hver av komponentene blir $ -1 $ når $ t=-1 $, for dette valget av $ t $ er derfor $ \vec{a}||\vec{b} $.

\opr{finnsogt}
$\vec{u}=[4, 6+s, -(s+t)] $ og $ \vec{v}=\left[\frac{12}{7}, \frac{2t-9s}{7}, \frac{3s-t}{7}\right] $
Vi starter med å observere at:
\[ \vec{v}=\frac{1}{7}[12, 2t-9s, 3s-t] \]
Vi definerer $ \vec{w}= [12, 2t-9s, 3s-t] $. Skal vi ha at $ \vec{u}||\vec{v} $, må vi også hat at $ \vec{u}||\vec{w} $. Siden forholdet mellom førstekomponentene til $ \vec{u} $ og $ \vec{w} $ er 3, krever vi at $ \vec{u}=3\vec{w} $. Da kan vi sette opp følgende ligningssystem:
\alg{
2t-9s &= 3(6+s) \tag{I}\label{3I}\br
3s-t &= -3(s+t) \tag{II}\label{3II}
}
Av \ref{3II} får vi at:
\alg{
3s-t &= -3s-3t \\
2t &= -6s \\
t &= -3s 
}
Setter vi $ t=-3s $ inn i (\ref{3II}) får vi:
\alg{
2(-3s) -9s &= 18+3s\\
-6s -9s &=  18 +3s \\
-18s &= 18 \\
s&=-1
}
Altså er $ \vec{u} $ og $ \vec{v} $ parallelle hvis $ s=-1 $ og $ t=-3s=3 $. 

\opr{vis22det}
\alg{
\left|\begin{matrix}
	ae & be \\
	cf & df
\end{matrix}\right|&= aedf-becf \\
&= ef(ad-bc) \\
&= ef\left|\begin{matrix}
	a & b \\
	c & d
\end{matrix}\right|
i}

\opr{arparo}

\textbf{b)} Arelet er gitt som tallverdien til $ \det(\vec{a}, \vec{b}) $:
\alg{
\det(\vec{a}, \vec{b}) &= \left|\begin{matrix}
	-2 & 4 \\
	24 & -16
\end{matrix}\right|\\	&= 2\cdot8 \left|\begin{matrix}
-1 & 2 \\
3 & -2
\end{matrix}\right| \\
&= 16((-1)\cdot(-2)-2\cdot3) \\
&= 16\cdot(-4) \\
&= -64
}
Arealet er altså 64.

\opr{abparvekpro0}\\
Hvis $ \vec{u}||\vec{v} $ betyr dette at hvis vi skriver $ \vec{u}=[a, b, c] $, så kan vi skrive $ \vec{v}=d[a, b, c] $. Vi får da at:
\alg{
\vec{u}\times\vec{v}&= \left|\begin{matrix}
	\vec{e}_x & \vec{e}_y & \vec{e}_z \\
	a & b & c \\
	da & db & dc
\end{matrix}\right| \\
&= d\vec{e}_x \left|\begin{matrix}
	b & c \\
	b & c
\end{matrix}\right|-d\vec{e}_y \left|\begin{matrix}
a & c \\
a & c
\end{matrix}\right|+d\vec{e}_x \left|\begin{matrix}
a & b \\
a & b
\end{matrix}\right| \\
&= 0
}
Resultatet fra \ref{vis22det} er her brukt i andre linje for å forenkle regningen av $ 2\times2 $ determinantene.

\opr{tetraro}\\
\textbf{a)} Arealet til grunnflaten tilsvarer lengden av vektoren $ \vec{a}\times\vec{b} $:
\alg{
	\vec{a}\times\vec{b}  &= 
\left|\begin{matrix}
	\vec{e}_x & \vec{e}_y & \vec{e}_z \\
	2 & -2 & 1 \\
	3 & -3 & 1
\end{matrix}\right| \\
&= [(-2)\cdot1-1\cdot(-3), -(2\cdot1-1\cdot3), 2\cdot(-3)-(-2)\cdot3] \\
&= [-2+3, -(2-3), -6+6] \\
&= [1, 1, 0] \\
& \\
|\vec{a}\times\vec{b} |&= \sqrt{1^2+1^2} \\
&= \sqrt{2}
}
\textbf{b)} Av (??) vet vi at volumet $ V $ er gitt som:
\[ V = \frac{1}{6}|\vec{a}\times\vec{b} \cdot \vec{c}| \]
Vi har at:
\algv{
\vec{a}\times\vec{b} \cdot \vec{c} &= [1, 1, 0]\cdot[2, -3, 2] \\
&= 2-3
&= -1
}
Og dermed er $ V=\frac{1}{6} $.

\opr{parfinnh}\\
\textbf{a)} Diagonalen til grunnflaten kan uttrykkes som vektoren $ \vec{a}+\vec{b} $, og lengden blir da (husk at $ |\vec{u}|^2 = \vec{u}^{\,2} $):
\alg{
|\vec{a}+\vec{b}| &= \sqrt{\left(\vec{a}+\vec{b}\right)^2} \\
&= \sqrt{\vec{a}^{\,2}+\vec{a}\cdot\vec{b}+\vec{b}^{\,2}} \\
&= \sqrt{3^2 +0 +4^2} \\
&= \sqrt{25} \\
&= 5
}
\end{document}