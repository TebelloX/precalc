%\documentclass[english, 11 pt, class=article, crop=false]{standalone}
%%\documentclass[english, 11 pt]{report}
\usepackage[T1]{fontenc}
\usepackage[utf8]{luainputenc}
\usepackage{babel}
\usepackage[hidelinks, bookmarks]{hyperref}
\usepackage{geometry}
\geometry{verbose,tmargin=1cm,bmargin=3cm,lmargin=4cm,rmargin=4cm,headheight=3cm,headsep=1cm,footskip=1cm}
\setlength{\parindent}{0bp}
\usepackage{amsmath}
\usepackage{amssymb}
\usepackage{esint}
\usepackage{import}
\usepackage[subpreambles=false]{standalone}
%\makeatletter
\addto\captionsenglish{\renewcommand{\chaptername}{Kapittel}}
\makeatother
\usepackage{tocloft}
\addto\captionsenglish{\renewcommand{\contentsname}{Innhold}}
\usepackage{graphicx}
\usepackage{placeins}
\raggedbottom
\usepackage{calc}
\usepackage{cancel}
\makeatletter
\usepackage{color}
\definecolor{shadecolor}{rgb}{0.105469, 0.613281, 1}
\usepackage{framed}
\usepackage{wrapfig}
\usepackage{bm}
\usepackage{ntheorem}

\usepackage{ragged2e}
\RaggedRight
\raggedbottom
\frenchspacing

\newcounter{lign}[section]
\newenvironment{lign}[1][]{\Large \refstepcounter{lign} \large
	\textbf{\thelign #1} \rmfamily}{\par\medskip}
\numberwithin{lign}{section}
\numberwithin{equation}{section}
\usepackage{xcolor}
\usepackage{icomma}
\usepackage{mathtools}
\usepackage{lmodern} % load a font with all the characters
\usepackage{xr-hyper}
\makeatother
\usepackage[many]{tcolorbox}

%\setlength{\parskip}{\medskipamount}
\newcommand{\parskiplength}{11pt}
%\setlength{\parskip}{0 pt}
\newcommand\eks[2][]{\begin{tcolorbox}[enhanced jigsaw,boxrule=0.3 mm, arc=0mm,breakable,colback=green!30] {\large \textbf{Eksempel #1} \vspace{\parskiplength}\\} #2 \vspace{1pt} \end{tcolorbox}\vspace{1pt}}

\newcommand\fref[2][]{\hyperref[#2]{\textsl{Figur \ref*{#2}#1}}}
\newcommand{\hr}[2]{\hyperref[#2]{\color{blue}\textsl{#1}}}

\newcommand\rgg[2][]{\begin{tcolorbox}[boxrule=0.3 mm, arc=0mm,colback=orange!55] #2 \vspace{1pt} \end{tcolorbox}\vspace{-2pt}}
\newcommand\alg[1]{\begin{align*} #1 \end{align*}}
\newcommand\algv[1]{\vspace{-11 pt} \begin{align*} #1 \end{align*}}
\newcommand\vs{\vspace{-11 pt}}
\newcommand\g[1]{\begin{center} {\tt #1}  \end{center}}
\newcommand\gv[1]{\begin{center} \vspace{-22 pt} {\tt #1} \vspace{-11 pt} \end{center}}
%\addto\captionsenglish{\renewcommand{\contentsname}{Løsningsforslag tentamen R2 H2015}}

% Farger
\colorlet{shadecolor}{blue!30} 

% Figur
\usepackage{float}
\usepackage{subfig}
\captionsetup[subfigure]{labelformat=empty}
\usepackage{esvect}

\newcommand\sv{\textbf{Svar:} \vspace{5 pt} \\}

%Tableofconents
\renewcommand{\cfttoctitlefont}{\Large\bfseries}
\setlength{\cftsubsecindent}{2 cm}
\newcommand\tocskip{6 pt}
\setlength{\cftaftertoctitleskip}{30 pt}
\setlength{\cftbeforesecskip}{\tocskip}
%\setlength{\cftbeforesubsecskip}{\tocskip}

%Footnote:
\usepackage[bottom, hang, flushmargin]{footmisc}
\usepackage{perpage} 
\MakePerPage{footnote}
\addtolength{\footnotesep}{2mm}
\renewcommand{\thefootnote}{\arabic{footnote}}
\renewcommand\footnoterule{\rule{\linewidth}{0.4pt}}

%asin, atan, acos
\DeclareMathOperator{\atan}{atan}
\DeclareMathOperator{\acos}{acos}
\DeclareMathOperator{\asin}{asin}

%Tabell
\addto\captionsenglish{\renewcommand{\tablename}{Figur}}

% Figur
\usepackage[font=footnotesize,labelfont=sl]{caption}
\addto\captionsenglish{\renewcommand{\figurename}{Figur}}

% Figurer
\newcommand\scr[1]{/home/sindre/R/scr/#1}
\newcommand\asym[1]{/home/sindre/R/asymptote/#1}

%Toc for seksjoner
\newcommand\tsec[1]{\phantomsection\addcontentsline{toc}{section}{#1}
	\section*{#1}}
%\newcommand\tssec[1]{\subsection*{#1}\addcontentsline{toc}{subsection}{#1}}
\newcommand\tssec[1]{\subsection*{#1}}
% GeoGebra
\newcommand{\cms}[2]{{\tt #1( #2 )}}
\newcommand{\cm}[2]{{\large \tt #1( #2 )} \gvs \\}
\newcommand{\cmc}[2]{{\large \tt #1( #2 )} \large (CAS)  \gvs \\ \normalsize}
\newcommand{\cmk}[2]{{\large \tt #1( #2 )} \large (Inntastingsfelt)  \gvs \\ \normalsize}

\newcommand\gvs{\vspace{11 pt}}

\newcommand\vsk{\vspace{11 pt}}
\newcommand{\merk}{\vsk \textsl{Merk}: }
\newcommand{\fig}[1]{
\begin{figure}
	\centering
	\includegraphics[scale=0.5]{fig/#1}
\end{figure}
}
\newcommand{\figc}[1]{
		\centering
		\includegraphics[scale=0.5]{fig/#1}
}

% Opg
%\newcommand{\opgt}{\phantomsection \addcontentsline{toc}{section}{Oppgaver} \section*{Oppgaver for kapittel \thechapter}}
\newcounter{opg}
\numberwithin{opg}{section}

\newcommand{\opl}[1]{\vspace{15pt} \refstepcounter{opg} \textbf{\theopg} \vspace{2 pt} \label{#1} \\}

\\
\documentclass[english, 11 pt, class=article, crop=false]{standalone}
%\documentclass[english, 11 pt]{report}
\usepackage[T1]{fontenc}
\usepackage[utf8]{luainputenc}
\usepackage{babel}
\usepackage[hidelinks, bookmarks]{hyperref}
\usepackage{geometry}
\geometry{verbose,tmargin=1cm,bmargin=3cm,lmargin=4cm,rmargin=4cm,headheight=3cm,headsep=1cm,footskip=1cm}
\setlength{\parindent}{0bp}
\usepackage{amsmath}
\usepackage{amssymb}
\usepackage{esint}
\usepackage{import}
\usepackage[subpreambles=false]{standalone}
%\makeatletter
\addto\captionsenglish{\renewcommand{\chaptername}{Kapittel}}
\makeatother
\usepackage{tocloft}
\addto\captionsenglish{\renewcommand{\contentsname}{Innhold}}
\usepackage{graphicx}
\usepackage{placeins}
\raggedbottom
\usepackage{calc}
\usepackage{cancel}
\makeatletter
\usepackage{color}
\definecolor{shadecolor}{rgb}{0.105469, 0.613281, 1}
\usepackage{framed}
\usepackage{wrapfig}
\usepackage{bm}
\usepackage{ntheorem}

\usepackage{ragged2e}
\RaggedRight
\raggedbottom
\frenchspacing

\newcounter{lign}[section]
\newenvironment{lign}[1][]{\Large \refstepcounter{lign} \large
	\textbf{\thelign #1} \rmfamily}{\par\medskip}
\numberwithin{lign}{section}
\numberwithin{equation}{section}
\usepackage{xcolor}
\usepackage{icomma}
\usepackage{mathtools}
\usepackage{lmodern} % load a font with all the characters
\usepackage{xr-hyper}
\makeatother
\usepackage[many]{tcolorbox}

%\setlength{\parskip}{\medskipamount}
\newcommand{\parskiplength}{11pt}
%\setlength{\parskip}{0 pt}
\newcommand\eks[2][]{\begin{tcolorbox}[enhanced jigsaw,boxrule=0.3 mm, arc=0mm,breakable,colback=green!30] {\large \textbf{Eksempel #1} \vspace{\parskiplength}\\} #2 \vspace{1pt} \end{tcolorbox}\vspace{1pt}}

\newcommand\fref[2][]{\hyperref[#2]{\textsl{Figur \ref*{#2}#1}}}
\newcommand{\hr}[2]{\hyperref[#2]{\color{blue}\textsl{#1}}}

\newcommand\rgg[2][]{\begin{tcolorbox}[boxrule=0.3 mm, arc=0mm,colback=orange!55] #2 \vspace{1pt} \end{tcolorbox}\vspace{-2pt}}
\newcommand\alg[1]{\begin{align*} #1 \end{align*}}
\newcommand\algv[1]{\vspace{-11 pt} \begin{align*} #1 \end{align*}}
\newcommand\vs{\vspace{-11 pt}}
\newcommand\g[1]{\begin{center} {\tt #1}  \end{center}}
\newcommand\gv[1]{\begin{center} \vspace{-22 pt} {\tt #1} \vspace{-11 pt} \end{center}}
%\addto\captionsenglish{\renewcommand{\contentsname}{Løsningsforslag tentamen R2 H2015}}

% Farger
\colorlet{shadecolor}{blue!30} 

% Figur
\usepackage{float}
\usepackage{subfig}
\captionsetup[subfigure]{labelformat=empty}
\usepackage{esvect}

\newcommand\sv{\textbf{Svar:} \vspace{5 pt} \\}

%Tableofconents
\renewcommand{\cfttoctitlefont}{\Large\bfseries}
\setlength{\cftsubsecindent}{2 cm}
\newcommand\tocskip{6 pt}
\setlength{\cftaftertoctitleskip}{30 pt}
\setlength{\cftbeforesecskip}{\tocskip}
%\setlength{\cftbeforesubsecskip}{\tocskip}

%Footnote:
\usepackage[bottom, hang, flushmargin]{footmisc}
\usepackage{perpage} 
\MakePerPage{footnote}
\addtolength{\footnotesep}{2mm}
\renewcommand{\thefootnote}{\arabic{footnote}}
\renewcommand\footnoterule{\rule{\linewidth}{0.4pt}}

%asin, atan, acos
\DeclareMathOperator{\atan}{atan}
\DeclareMathOperator{\acos}{acos}
\DeclareMathOperator{\asin}{asin}

%Tabell
\addto\captionsenglish{\renewcommand{\tablename}{Figur}}

% Figur
\usepackage[font=footnotesize,labelfont=sl]{caption}
\addto\captionsenglish{\renewcommand{\figurename}{Figur}}

% Figurer
\newcommand\scr[1]{/home/sindre/R/scr/#1}
\newcommand\asym[1]{/home/sindre/R/asymptote/#1}

%Toc for seksjoner
\newcommand\tsec[1]{\phantomsection\addcontentsline{toc}{section}{#1}
	\section*{#1}}
%\newcommand\tssec[1]{\subsection*{#1}\addcontentsline{toc}{subsection}{#1}}
\newcommand\tssec[1]{\subsection*{#1}}
% GeoGebra
\newcommand{\cms}[2]{{\tt #1( #2 )}}
\newcommand{\cm}[2]{{\large \tt #1( #2 )} \gvs \\}
\newcommand{\cmc}[2]{{\large \tt #1( #2 )} \large (CAS)  \gvs \\ \normalsize}
\newcommand{\cmk}[2]{{\large \tt #1( #2 )} \large (Inntastingsfelt)  \gvs \\ \normalsize}

\newcommand\gvs{\vspace{11 pt}}

\newcommand\vsk{\vspace{11 pt}}
\newcommand{\merk}{\vsk \textsl{Merk}: }
\newcommand{\fig}[1]{
\begin{figure}
	\centering
	\includegraphics[scale=0.5]{fig/#1}
\end{figure}
}
\newcommand{\figc}[1]{
		\centering
		\includegraphics[scale=0.5]{fig/#1}
}

% Opg
%\newcommand{\opgt}{\phantomsection \addcontentsline{toc}{section}{Oppgaver} \section*{Oppgaver for kapittel \thechapter}}
\newcounter{opg}
\numberwithin{opg}{section}

\newcommand{\opl}[1]{\vspace{15pt} \refstepcounter{opg} \textbf{\theopg} \vspace{2 pt} \label{#1} \\}



\begin{document}
\eqlen	
\opgt

\setcounter{section}{1}	
\opl{leno}
Finn lengden av vektorene:\os
\begin{tabular}{@{}l l}
\textbf{a)} $ [-2, 1, 5] $ & \quad\textbf{b)} $ [\sqrt{3}, 2,  \sqrt{2}] $
\end{tabular}	
	
\opl{closest}
Hvilket av punktene $ {B=(3, -2, 1)} $ og $ {C=(0, 5, 6) }$ ligger nærmest punktet $ {A=(1, -1, -2)} $?

\opl{lenfakt}
Gitt vektoren
\[ \vec{u}=[ad, bd, cd]\]
\textbf{a)} Vis at
\[ |\vec{u}|=d\sqrt{a^2 + b^2 + c^2} \]
når $ d>0 $.\os

\textbf{b)} Forklar at
\[ |\vec{u}|=|d|\sqrt{a^2 + b^2 + c^2} \]
når $ d<0 $.

\nes
\opl{skalfaktor}
Gitt vektorene
\[ \vec{u}=[ad, bd, cd] \text{ og } \vec{v}=[eh, fh, gh] \]
Vis at
\[ \vec{u}\cdot\vec{v}=dh(ae+bf+cg) \] \vds

\opl{skalproo}
Finn skalarproduktet av vektorene:\os

\textbf{a)} $ \vec{a}=[2, 4, 6] $ og $ \vec{b}=[-5, 0, -1] $\os 

\textbf{b)} $ \vec{a}=[-9, 1, 5] $ og $\vec{b}= [-2, 1, -2] $\os

\textbf{c)} $ \vec{a}=\left[\frac{1}{5}, \frac{3}{5},- \frac{1}{5}\right] $ og $ \vec{b}=[512, -128, 64] $. \textsl{Tips:} Bruk resultatet fra opg. \ref{skalfaktor}.
\newpage
\opl{skalpro2o}
Finn skalarproduktet av $ \vec{a} $ og $ \vec{b} $, som utspenner vinkelen $ \theta $, når du vet at\os

\textbf{a)} $ |\vec{a}|=5 $, $ |\vec{b}|= 2$ og $ \theta = 60^\circ $\os

\textbf{b)} $ |\vec{a}|=5 $, $ |\vec{b}|= 2$ og $ \theta = 150^\circ $

\opl{finntheta} 
Finn vinkelen mellom $ \vec{a} $ og $ \vec{b} $ når\os

\textbf{a)} $ \vec{a}=[ 5 ,-5,  2]$ og $ \vec{b}=[ 3 ,-4 , 5] $\os

\textbf{b)} $ \vec{a}=[ 2 ,-1,  -3]$ og $ \vec{b}=[ -1 ,-3 , -2] $\os

\textbf{c)} $ \vec{a}=[ -1 ,-2,  2]$ og $ \vec{b}=[ -3 , 5 , -4] $

\opl{forkort}
Forkort uttrykkene når du vet at $ |\vec{a}|=1 $, $ |\vec{b}|=2 $, $ |\vec{c}|=5 $, $ \vec{a}\cdot\vec{b}=0 $ og $ \vec{b}\cdot\vec{c}=0 $.\os

\textbf{a)} $ \vec{b}\cdot(\vec{a}+\vec{c}) + 3(\vec{a}+\vec{b})^2 $ \os

\textbf{b)} $ (\vec{a}+ \vec{b}+\vec{c})^2$

\nes
\opl{orto}
Sjekk om $ \vec{a} $ og $ \vec{b} $ er ortogonale når\os

\textbf{a)} $ \vec{a}=[2, 4, -2] $ og $ \vec{b}=[3, 1, 1] $\os

\textbf{b)} $ \vec{a}=[-18, 12, 9] $ og $ \vec{b}=[1, -2, 1] $\os

\textbf{c)} $ \vec{a}=[5, 5, -1] $ og $ \vec{b}=[5, -4, 5] $

\opl{torto}
Gitt vektoren
\[ \vec{u}=[-5, -1, 6] \]
Finn $ t $ slik at $ \vec{u}\perp \vec{v} $ når\os

\textbf{a)} $ \vec{v}=[t, 3t, 2] $\os

\textbf{b)} $ \vec{v}=[t, t^2, 1] $

\opl{sjekkpar}
Sjekk om $ \vec{a}\parallel\vec{b} $ når\os

\textbf{a)} $ \vec{a}=[8, 4, -2] $ og $ \vec{b}=[4, 2, 4] $\os

\textbf{b)} $ \vec{a}=[-3, 5, 2] $ og $ \vec{b}=\left[-\frac{9}{7}, \frac{15}{7}, \frac{6}{7}\right] $ 
\newpage
\opl{finntpar}
Gitt vektoren 
\[ \vec{a}=[-3, 1, 8] \]
Om mulig, finn $ t $ slik at $ \vec{a}\parallel\vec{b} $ når\os

\textbf{a)} $ \vec{b}=[t+3, 1-t, -16] $\os

\textbf{b)} $ \vec{b}=[t^2+2, t, -(5t^2+3)] $

\opl{finnsogt}
Finn $ s $ og $ t $ slik at $\vec{u}=[4, 6+s, -(s+t)] $ og $ \vec{v}=\left[\frac{12}{7}, \frac{2t-9s}{7}, \frac{3s-t}{7}\right] $ er parallelle. \os

\nes
\opl{vis22det}
Vis at
\[  \left|\begin{matrix}
ae & be \\
cf & df
\end{matrix}\right|=ef\left|\begin{matrix}
a & b \\
c & d
\end{matrix}\right| \]

\begin{comment}
\opl{arparo}
Finn aralet til parallellogrammet utspent av (\textsl{Tips:} Bruk resultatet fra opg. \ref{vis22det}):

\textbf{a)} $ [-2, 7] $ og $ [12, 8] $

\textbf{b)} $ [-2, 4] $ og $ [24, -16] $\\
\end{comment}

\opl{abparvekpro0}
Vis at hvis $ \vec{u}||\vec{v} $, så er $ \vec{u}\times\vec{v}=0 $

\opl{lagrangesid}
For to vektorer $ \vec{u} $ og $ \vec{v} $ er \textit{Lagranges identitet} gitt som
\[ |\vec{u}\times\vec{v}|^2=|\vec{u}|^2|\vec{v}|^2-(\vec{u}\cdot\vec{v}\,)^2 \]
Bruk identiteten og definisjonen av skalarproduktet til å vise at
\[ |\vec{u}\times\vec{v}|=|\vec{u}||\vec{v}|\sin \angle(\vec{u}, \vec{v})  \]\vds

\opl{tetraro}
Et tetraeted er utspent av vektorene $ \vec{a}=[2, -2, 1],\; \vec{b}=[3, -3, 1] $ og $ \vec{c}=[2, -3, 2] $, hvor $ \vec{a} $ og $ \vec{b} $ utspenner grunnflaten.\os

\textbf{a)} Vis at arealet av grunnflaten er $ \sqrt{2} $.\os

\textbf{b)} Vis at volumet av tetraetedet er $ \frac{1}{6} $.
\newpage
\opl{parfinnh}
Et parallellepidet er utspent av vektorene $ \vec{a},\; \vec{b} $ og $ \vec{c} $. Vi har at ${|a|=3}$,  ${\vec{b}|=4}$ og $ {\vec{a}\cdot \vec{b}=0}  $ og at grunnflaten er utspent av $ \vec{a} $ og $ \vec{b} $. \os

\textbf{a)} Finn lengden av diagonalen til grunnflaten.\os

La $ \theta $ være vinkelen mellom $ {\vec{a}\times\vec{b}} $ og $ \vec{c} $ og la $ {\theta\in[0^\circ, 90^\circ]} $.\os

\textbf{b)} Lag en tegning og forklar hvorfor høyden $ h $ i parallellepipedet er gitt som
\[ h= |\vec{c}|\cos \theta \]
\textbf{d)} Forklar hvorfor volumet $ V $ av parallellepidetet kan skrives som
\[ V= |\vec{a}\times\vec{b}||c|\cos \theta\]\vds

\opl{kryserlikdet}
Gitt vektorene $ \vec{u}=[a, b, c] $, $ \vec{v}=[d, e, f] $ og $ \vec{w}=[g, h, i] $. Vis at
\[ \vec{u}\times\vec{v}\cdot\vec{w}= \vec{w}\times\vec{u}\cdot\vec{v}\]
Tre pyramider er utspent av vektorene $ \vec{u}=[a, b, c] $, $ \vec{v}=[d, e, f] $ og $ \vec{w}=[g, h, i] $. Grunnflatene til pyramidene er henholdsvis utspent av $ \vec{u} $ og $ \vec{v} $, $ \vec{u} $ og $ \vec{w} $ og $ \vec{v} $ og $ \vec{w} $. Hva er uttrykket til volumet av pyramidene?

\ekspop
Vis at tallverdien til $ \det(\vec{u}, \vec{v}) $ tilsvarer arealet $ A $ av parallellogrammet (i planet) utspent av $ \vec{u}=[a, b] $ og $ \vec{v}=[c, d] $:
\[ A = |\det(\vec{u}, \vec{v})| \]
\subimport{}{ar1}

\end{document}