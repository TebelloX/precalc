%\documentclass[english, 11 pt]{report}
\usepackage[T1]{fontenc}
\usepackage[utf8]{luainputenc}
\usepackage{babel}
\usepackage{xr-hyper}

\usepackage{geometry}
%\geometry{verbose,margin=1.5cm,bmargin=2.5cm,lmargin=4.5cm,rmargin=4.5cm,headheight=10cm,headsep=1cm,footskip=1cm}
\geometry{verbose,paperwidth=16.1 cm, paperheight=24 cm, inner=2.3cm, outer=1.8 cm, bmargin=2cm, tmargin=1.8cm}
\setlength{\parindent}{0bp}
\usepackage{amsmath}
\usepackage{amssymb}
\usepackage{esint}
\usepackage{import}
\usepackage[subpreambles=false]{standalone}
%\makeatletter
\usepackage{tocloft}
\usepackage{graphicx}
\usepackage{placeins}
\usepackage{calc}
\usepackage{cancel}
\usepackage{color}
\definecolor{shadecolor}{rgb}{0.105469, 0.613281, 1}
\usepackage{framed}
\usepackage{wrapfig}
\usepackage{bm}
\usepackage{ntheorem}
\usepackage{ragged2e}
\RaggedRight
\raggedbottom
%\flushbottom
\frenchspacing

\newcounter{lign}[chapter]
\newenvironment{lign}[1][]{\Large \refstepcounter{lign} \large
	\textbf{\thelign #1} \rmfamily}{\par\medskip}
\numberwithin{lign}{chapter}
\numberwithin{equation}{chapter}
\usepackage{xcolor}
\usepackage{icomma}
\usepackage{mathtools}
\usepackage{lmodern} % load a font with all the characters
\makeatother
\usepackage[many]{tcolorbox}

\newcommand{\parskiplength}{11pt}
%\setlength{\parskip}{\parskiplength}

\newcommand\eks[2][]{\begin{tcolorbox}[arc=0mm,enhanced jigsaw,breakable,colback=green!8,boxrule=0.3 mm] {\large \textbf{Eksempel #1} \vspace{5 pt}\newline} #2 \vspace{1pt} \end{tcolorbox}\vspace{-4pt}}
\newcommand\tit[2][]{\large \textbf{#2 - Eksempel #1} \\}
\newcommand\lig[1]{\begin{lign} \large \normalfont #1 \end{lign}}
\newcommand\hrds[1]{\hyperref[#1]{\textsl{delseksjon \ref*{#1}}}}
\newcommand\hr[2][]{\hyperref[#2]{\textsl{#1}}}
\newcommand\hrs[2][]{\hyperref[#2]{\textsl{\textsl{#1} \ref*{#2}}}}
\newcommand\hrv[1]{\hyperref[#1]{\textsl{\textsl{vedlegg} \ref*{#1}}}}
\newcommand\fref[2][]{\hyperref[#2]{\textsl{figur \ref*{#2}#1}}}
\newcommand\hrss[2][]{\hyperref[#2]{\textsl{#1}}}
\newcommand\ligg[1]{{\large \normalfont \textbf{#1}} \vspace{5 pt}\\}
%\newcommand\rg[2][]{\begin{tcolorbox}[colback=blue!15] \ligg{#1} #2  %\end{tcolorbox}}
\newcommand\rg[2][]{\begin{tcolorbox}[arc=0mm,colback=blue!5,enhanced jigsaw,breakable, boxrule=0.3 mm]{\large \normalfont \textbf{#1} \vspace{5 pt}\newline} #2 \vspace{1pt} \end{tcolorbox}\vspace{-4pt}}
\newcommand\rgg[2][]{\begin{tcolorbox}[colback=orange!55] #2 \vspace{1pt} \end{tcolorbox}\vspace{-9pt}}
\newcommand\alg[1]{\begin{align*} #1 \end{align*}}
\newcommand{\algv}[1]{\vspace{-9 pt} \begin{align*} #1 \end{align*}}
\newcommand{\algb}[1]{\vspace{-9 pt} \begin{align*} #1 \end{align*}}
\newcommand\vs{\vspace{-\parskiplength}}
\newcommand\vsb{\vspace{-13pt}}
\newcommand\vds{\vs\vs}
\newcommand\g[1]{\begin{center} \vspace{-11 pt} {\tt #1} \vspace{-11 pt} \end{center}}
\newcommand\gv[1]{\begin{center} \vspace{-22 pt} {\tt #1} \vspace{-11 pt} \end{center}}
\newcommand\vsk{\vspace{11 pt}}
%\addto\captionsenglish{\renewcommand{\contentsname}{Løsningsforslag tentamen R2 H2015}}

% Farger
%\pagecolor{yellow!3}
\colorlet{shadecolor}{blue!30} 

% Figur
\usepackage{float}
\usepackage{subfig}
\captionsetup[subfigure]{labelformat=empty}
\newcommand{\fig}[2]{\begin{figure}
		\centering
		\includegraphics[]{\asym{#1}}
		\caption{#2}
\end{figure}}
\newcommand{\net}[2]{{\color{blue}\href{#1}{#2}}}

\usepackage{esvect}
\usepackage[font=footnotesize,labelfont=sl]{caption}
\addto\captionsenglish{\renewcommand{\figurename}{Figur}}

\newcommand{\sss}[1]{\subsection*{#1}   \addcontentsline{toc}{subsection}{#1}}
\newcommand\sv{\vsk \textbf{Svar:} \vspace{4 pt}\\}
\newcommand{\bs}{\\[4 pt]}
\newcommand{\os}{\\[4 pt]}
%Tableofconents
%\setlength{\cftsubsecindent}{1 cm}
\renewcommand{\cfttoctitlefont}{\Large\bfseries}
\setlength{\cftaftertoctitleskip}{0 pt}
\setlength{\cftbeforetoctitleskip}{0 pt}
\renewcommand\cftchapfont{\footnotesize\bfseries}
\renewcommand\cftsecfont{\footnotesize}
\renewcommand\cftsubsecfont{\footnotesize}
\addto\captionsenglish{\renewcommand{\contentsname}{Innhold}}
\addto\captionsenglish{\renewcommand{\chaptername}{Kapittel}}
\newcommand\tocskip{3 pt}
\setlength{\cftbeforechapskip}{12 pt}
\setlength{\cftbeforesecskip}{\tocskip}
\setlength{\cftbeforesubsecskip}{\tocskip}

%Seksjoner
\usepackage{titlesec}
\titleformat{\chapter}[display]
{\normalfont\LARGE\bfseries}{\chaptertitlename\ \thechapter}{20pt}{\Huge}
\titlespacing{\chapter}{0pt}{0pt}{0pt}
%\titlespacing{\subsection}{0pt}{\parskip}{0pt}
%\titlespacing{\section}{0pt}{\parskip}{0pt}

% Gjem tekst
\newcommand\gj[1]{\begin{comment} #1 \end{comment}}

%Footnote:
\usepackage[bottom, hang, flushmargin]{footmisc}
\usepackage{perpage} 
\MakePerPage{footnote}
\addtolength{\footnotesep}{2mm}
\renewcommand{\thefootnote}{\arabic{footnote}}
\renewcommand\footnoterule{\rule{\linewidth}{0.4pt}}

%asin, atan, acos
\DeclareMathOperator{\atan}{atan}
\DeclareMathOperator{\acos}{acos}
\DeclareMathOperator{\asin}{asin}

%Tabell
\addto\captionsenglish{\renewcommand{\tablename}{Tabell}}

% Tikz
\usetikzlibrary{calc}
\usepackage{tkz-euclide}
\tikzset{
	font={\fontsize{11pt}{12}\selectfont}}
\usepackage{pgfplots}
\usetikzlibrary{matrix}

%Nummererte ligninger
%\newcommand\nreq[1]{\begin{equation*} #1 \end{equation*}}
\newcommand\nreq[1]{\begin{equation} #1 \end{equation}}

%screenshots and asymptote
%\newcommand\scr[1]{/media/sindre/TR2/scr/#1}
%\newcommand\asym[1]{/media/sindre/TR2/R/asymptote/#1}
\newcommand{\scr}[1]{/home/sindre/R/scr/#1}
\newcommand{\asym}[1]{/home/sindre/R/asymptote/#1}
%Toc for seksjoner
\newcommand\tsec[1]{\phantomsection \addcontentsline{toc}{section}{#1}
	\section*{#1}}
%\newcommand\tssec[1]{\subsection*{#1}\addcontentsline{toc}{subsection}{#1}}
\newcommand\tssec[1]{\subsection{#1}}

% GeoGebra
\newcommand\cm[1]{{\large \tt #1} \gvs\\}
\newcommand\cmc[1]{{\large \tt #1} {\large (CAS)} \gvs\\}
\newcommand\cmk[1]{{\large \tt #1} {\large (Inntastingsfelt)} \gvs\\}
\newcommand\gvs{\vspace{\parskip}}

% Brok
\newcommand\br{\\[5 pt]}

% Opg
\newcommand{\opgt}{\phantomsection \addcontentsline{toc}{section}{Oppgaver} \section*{Oppgaver for kapittel \thechapter}}
\newcounter{opg}
\numberwithin{opg}{section}
\newcommand{\op}{\refstepcounter{opg} \textbf{\theopg} \vspace{2 pt} \\}
\newcommand{\opl}[1]{\vsk \refstepcounter{opg} \textbf{\theopg} \vspace{2 pt} \label{#1} \\}
\newcommand{\ekspop}{\vsk\textbf{Gruble \thechapter}\vspace{2 pt} \\}
\newcommand{\nes}{\stepcounter{section}
	\setcounter{opg}{0}}
\newcommand{\ness}{\stepcounter{subsection}
	\setcounter{opg}{0}}
\newcommand{\opr}[1]{\textbf{\ref{#1}}}
\newcommand{\se}[1]{Se eksempel s. {\pageref{#1}}}
\newcommand{\sel}{Se løsningsforslag.}

% Kolonner
%\usepackage{multicol}

%Vedlegg
\newcounter{vedl}
\newcounter{vedleq}
\renewcommand\thevedl{\Alph{vedl}}	
\newcommand{\vedlegg}[1]{\refstepcounter{vedl}\section*{Vedlegg \thevedl: #1}  \setcounter{vedleq}{0}}
\newcommand{\nreqvd}{\refstepcounter{vedleq}\tag{\thevedl \thevedleq}}

%page number
\usepackage{fancyhdr}
\pagestyle{fancy}
\fancyhf{}
\renewcommand{\headrule}{}
\fancyhead[RO, LE]{\thepage}

%more spaces
\newcommand{\regv}{\vspace{5pt}}

%equation
\newcommand{\y}[1]{$ {#1} $}

% index
\usepackage{imakeidx}
\makeindex[title=Indeks]

%fotnoter i tekstbokser med arabiske nummer
\renewcommand{\thempfootnote}{\arabic{mpfootnote}}

\usepackage[]{hyperref}

\newcommand{\eqlen}{
	%\setlength\abovedisplayskip{8pt plus 1mm minus 1mm}
	%	\setlength\belowdisplayskip{8pt plus 3pt minus 6pt}
	%	\setlength\abovedisplayshortskip{0pt plus 3pt}
	%	\setlength\belowdisplayshortskip{8pt plus 3.5pt minus 3pt}
	%\setlength\abovedisplayskip{8pt plus 0mm minus 0mm}
	%\setlength\belowdisplayskip{8pt plus 0pt minus 0pt}
	%\setlength\abovedisplayshortskip{0pt plus 0pt minus 0pt}
	%\setlength\belowdisplayshortskip{8pt plus 0pt minus 0pt}
	\allowdisplaybreaks
}

\usepackage{datetime2}
